\documentclass[12pt, answers]{exam}     % 试卷
\usepackage{ctex}          % 中文支持
\usepackage{amsmath, amssymb, amsthm}       % 数学公式包
\usepackage{enumitem}
\usepackage{geometry}      % 设置页边距 
\geometry{a4paper, margin = 1.5cm}

% 定义环境  
\newcommand{\anothersolution}{\par\noindent\textbf{另解:}}

% 自定义命令
\newcommand{\rank}{\operatorname{rank}}
\newcommand{\tr}{\operatorname{tr}}
\newcommand{\diag}{\operatorname{diag}}
\newcommand{\R}{\mathbb{R}}
\newcommand{\C}{\mathbb{C}}
\newcommand{\A}{\mathcal{A}}
\newcommand{\T}{\mathrm{T}}

% 标题部分  
\title{数值线代の题库三}
\author{by 23大数据miracle}
\date{}
\begin{document}

\maketitle

\begin{questions}

% =============== 题1 ===============
\question{} 证明Gershgorin圆盘定理: 
设$ A \in \C^{n \times n} $, $ A = (a_{ij}) $, 令
\[
G_i(A) = \{ z \in \C : |z - a_{ii}| \leq \sum_{j \neq i} |a_{ij}| \}
\]
其中 $i = 1, 2, \ldots, n $, 则有
\[ \lambda(A) \subset G_1(A) \cup G_2(A) \cup \ldots \cup G_n(A)
\]

\begin{solution}
设 \(\lambda\) 是 \(A\) 的特征值, \(x = (x_1, x_2, \dots, x_n)^{\T} \in \C^n \setminus \{ 0 \}\) 是对应的特征向量, 满足 \(A x = \lambda x\). 
取下标 \(r\) 使得
\[
|x_r| = \max_{1 \leq k \leq n} \{ |x_k| \}
\]
因 \(x \neq 0 \), 有 \(|x_r| > 0 \). 

考虑 \(A x = \lambda x\) 展开后的第 \(r\) 个分量: 
\[
\sum_{j=1}^n a_{rj} x_j = \lambda x_r
\]
移项得
\[
(\lambda - a_{rr}) x_r = \sum_{j \neq r} a_{rj} x_j
\]
取绝对值并应用三角不等式: 
\[
|\lambda - a_{rr}| \cdot |x_r| = \left| \sum_{j \neq r} a_{rj} x_j \right| \leq \sum_{j \neq r} |a_{rj}| \cdot |x_j|
\]
又 \(|x_j| \leq |x_r|\) 对所有 \(j\) 成立, 故
\[
\sum_{j \neq r} |a_{rj}| \cdot |x_j| \leq \sum_{j \neq r} |a_{rj}| \cdot |x_r| = |x_r| \sum_{j \neq r} |a_{rj}|
\]
代入原不等式, 得
\[
|\lambda - a_{rr}| \cdot |x_r| \leq |x_r| \sum_{j \neq r} |a_{rj}|
\]
两边除以 \(|x_r| > 0 \): 
\[
|\lambda - a_{rr}| \leq \sum_{j \neq r} |a_{rj}|
\]
因此, \(\lambda \in G_r \subset G_1(A) \cup G_2(A) \cup \ldots \cup G_n(A) \), 由于 \( \lambda \) 的任意性 , 命题得证. 

\begin{anothersolution}
反证法. 假设存在 $ A $ 的特征值 $\lambda^*$ 满足 $\lambda^* \notin \bigcup_{i=1}^n G_i(A)$, 即
\[
|\lambda^* - a_{ii}| > \sum_{j \neq i} |a_{ij}|, \quad \forall i = 1, 2, \ldots, n. 
\]
考虑矩阵 $B = \lambda^* I - A$, 则一定有 $ \det(B) = 0 $ , 且又由假设得对于任意的 $ i $ 都有
\[
|b_{ii}| = |\lambda^* - a_{ii}| > \sum_{j \neq i} |a_{ij}| = \sum_{j \neq i} |b_{ij}|
\]
故 $ B $ 严格对角占优. 而严格对角占优矩阵非奇异, 则 $ \det(B) \ne 0 $, 矛盾. 因此假设不成立, 原命题得证.
\end{anothersolution}

\end{solution}

% =============== 题2 =============== 
\question{}设 \( A \) 是 \( n \times m \) 矩阵, \( B \) 是 \( m \times n \) 矩阵, 且 \( m \geq n \). 证明: 
\[
\lambda(BA) = \lambda(AB) \cup \{\underbrace{0, \ldots, 0}_{\text{$m-n$ 个}}\}
\]

\begin{solution}
直接应用降阶公式(按两种方式化为分块三角阵): 
\[
\det\begin{bmatrix}
\lambda I_n & -A \\
-B & I_m
\end{bmatrix} = \det(\lambda I_n - AB) = \lambda^{n-m}\det(\lambda I_m - BA)
\]

由此得到特征多项式关系: 
\[
\det(\lambda I_m - BA) = \lambda^{m-n} \det(\lambda I_n - AB)
\]

因此, 直接有特征值集合满足: 
\[
\lambda(BA) = \lambda(AB) \cup \{\underbrace{0, \ldots, 0}_{m-n \text{个}}\}
\]
\end{solution}


% =============== 题3 =============== 
\question{}设 \( A = \begin{bmatrix} \alpha & \gamma \\ 0 & \beta \end{bmatrix} \), \(\alpha \neq \beta\), 
求 \(A\) 的特征值 \(\alpha\) 和 \(\beta\) 的条件数. 

\begin{solution}
由单特征值的条件数定义, 即:
\[
\operatorname{cond}(\lambda) = \|y\|_{2}
\]
其中 $ y $ 满足 $ y^{\T}A = \lambda y^{\T} $ 且 $ y^{\T}x = 1 $, \(x\) 为 \(A\) 对应特征值 $ \lambda $ 的单位特征向量, 可知

\begin{enumerate}
    \item 对于特征值 \(\alpha\):
    \[
    A - \alpha I = \begin{bmatrix} 
    0 & \gamma \\ 
    0 & \beta - \alpha 
    \end{bmatrix}
    \]
    
    自然有单位右特征向量:
    \[ 
    x_1 = \begin{bmatrix} 1 \\ 0 \end{bmatrix}
    \]
    
    又有:
    \[
    A^{\T} - {\alpha} I = \begin{bmatrix} 
    0 & 0 \\ 
    {\gamma} & {\beta} - {\alpha} 
    \end{bmatrix}
    \]
    
    取满足与 $ x_1 $ 内积为1的左特征向量 \( y_1 \), 满足:
    \[ 
    y_1 = \begin{bmatrix} 1 \\ \frac{\gamma}{\alpha - \beta} \end{bmatrix}
    \]
    
    则 $ \alpha $ 的条件数:
    \[
    \operatorname{cond}(\alpha) = \|y_1\|_{2} = \dfrac{\sqrt{|\alpha - \beta|^2 + |\gamma|^2}}{|\alpha - \beta|}
    \]
    
    \item 对于特征值 \(\beta\): 
    \[
    A - \beta I = \begin{bmatrix} 
    \alpha - \beta & \gamma \\ 
    0 & 0 
    \end{bmatrix}
    \]
    
    自然有单位右特征向量:
    \[
    x_2 = \frac{1}{\sqrt{|\gamma|^2 + |\beta - \alpha|^2}} \begin{bmatrix} \gamma \\ \beta - \alpha \end{bmatrix}
    \]
    
    又有:
    \[
    A^{\T} - {\beta} I = \begin{bmatrix} 
    {\alpha} - {\beta} & 0 \\ 
    {\gamma} & 0 
    \end{bmatrix}
    \]
    
    取满足与 $ x_2 $ 内积为1的左特征向量 \( y_2 \), 满足:
    \[ 
    y_2 = \dfrac{1}{\beta - \alpha} \begin{bmatrix} 0 \\ \sqrt{|\gamma|^2 + |\beta - \alpha|^2} \end{bmatrix}
    \]
    
    则 $ \beta $ 的条件数:
    \[
    \operatorname{cond}(\beta) = \| y \|_2 = \frac{\sqrt{|\beta - \alpha|^2 + |\gamma|^2}}{|\beta - \alpha|}
    \]

\end{enumerate}

\end{solution}


% =============== 题4 =============== 
\question{}证明特征值和特征向量的条件数在酉相似条件下保持不变. 

\begin{solution}
单位特征向量的条件数定义(单特征值的条件数定义见上题), 即:
\[
\operatorname{cond}(x) = \| \Sigma^{\perp} \|_2 = \| U_2 ( \lambda I - A_2 )^{-1} U_2^* \|_2
\]
其中, $U^* A U = \begin{bmatrix} \lambda & * \\ 0 & A_2 \end{bmatrix} $, $ U = \begin{bmatrix} x & U_2 \end{bmatrix} $. 
设$ A $的右、左特征向量分别是$ x $和$ y^{\T}$, 再令 $ B = W^* A W $, $ W $ 满足 $ W^*W = I $, 于是有
\begin{enumerate}
    \item 特征值条件数的不变性:
    
    \(B\) 的特征值仍为 \(\lambda\). 令 \(x_B = W^* x\), 则 \(B x_B = \lambda x_B\) 且 \(\|x_B\|_2 = 1\). 令 \(y_B = W^{\T} y\), 则
    \[
    y_B^{\T} B = y_B^{\T} (W^* A W)  = y^{\T} A W = \lambda y^{\T} W = \lambda y_B^{\T}
    \]
    且
    \[
    y_B^{\T} x_B = (W^{\T} y)^{\T} (W^{*} x) = y^{\T} x = 1
    \]
    因此, 
    \[
    \operatorname{cond}_B(\lambda) = \|y_B\|_2 = \|W^{\T} y\|_2 = \|y\|_2 = \operatorname{cond}_A(\lambda)
    \]
    
    \item 特征向量条件数的不变性:
    
    令 \(U_B = W^* U = \begin{bmatrix} W^* x & W^* U_2 \end{bmatrix} = \begin{bmatrix} x_B & U_{2, B} \end{bmatrix}\), 其中 \(U_{2, B} = W^* U_2\). 则
    \[
    U_B^* B U_B = (W^* U)^* (W^* A W) (W^* U) = U^* A U = \begin{bmatrix} \lambda & * \\ 0 & A_2 \end{bmatrix}
    \]
    故 \(B_2 = A_2\). 则特征向量条件数为
    \[
    \operatorname{cond}_B(x_B) = \| U_{2, B} (\lambda I - B_2)^{-1} U_{2, B}^* \|_2 = \| W^* U_2 (\lambda I - A_2)^{-1} U_2^* W \|_2
    \]
    由于谱范数在酉变换下不变, 有
    \[
    \operatorname{cond}_B(x_B) = \| U_2 (\lambda I - A_2)^{-1} U_2^* \|_2 = \operatorname{cond}_A(x)
    \]
\end{enumerate}
综上, \(\operatorname{cond}(\lambda)\) 和 \(\operatorname{cond}(x)\) 在酉相似变换下均保持不变. 

\end{solution}


% =============== 题5 =============== 
\question{}在幂法中, 取 \( A = \begin{bmatrix} 1 & 1 & 0 \\ 0 & 1 & 1 \\ 0 & 0 & 1 \end{bmatrix} \), \( u_0 = (0, 0, 1)^{\T} \). 
得到一个精确到 5 位有效数字的特征向量需要多少次迭代?

\begin{solution}注意到有分解 \( A = I + J_3(0) \), 其中
\[
J_3(0) = \begin{bmatrix} 0 & 1 & 0 \\ 0 & 0 & 1 \\ 0 & 0 & 0 \end{bmatrix}
\]

而又
\[
(J_3(0))^2 = \begin{bmatrix} 0 & 0 & 1 \\ 0 & 0 & 0 \\ 0 & 0 & 0 \end{bmatrix}
\]
且 \( (J_3(0))^k = 0 \), \( \forall k \geq 3 \). 

由二项式定理(又单位矩阵和任意矩阵都可交换)有
\[
A^n = (I + J_3(0))^n = \binom{n}{0} I + \binom{n}{1} J_3(0) + \binom{n}{2} (J_3(0))^2
\]
代入得
\[
A^n = \begin{bmatrix} 1 & n & \frac{n(n-1)}{2} \\ 0 & 1 & n \\ 0 & 0 & 1 \end{bmatrix}
\]

于是有
\[
v_n = A^n u_0 = \begin{bmatrix} 1 & n & \frac{n(n-1)}{2} \\ 0 & 1 & n \\ 0 & 0 & 1 \end{bmatrix} \begin{bmatrix} 0 \\ 0 \\ 1 \end{bmatrix} = \begin{bmatrix} \frac{n(n-1)}{2} \\ n \\ 1 \end{bmatrix}
\]

又 \( \| v_n \|_{\infty} =\frac{n(n-1)}{2}, \forall n \ge 3 \), 根据幂法迭代公式, 
\[
u_n = \frac{v_n}{\| v_n \|_{\infty}} = \begin{bmatrix} 1 \\ \frac{2}{n-1} \\ \frac{2}{n(n-1)} \end{bmatrix}
\]

事实上, 由于这种情况下可以收敛到 $ A $ 的精确特征向量为 \( \begin{bmatrix} 1 \\ 0 \\ 0 \end{bmatrix} \), \\
则精确到 5 位有效数字需满足: \\
\[
\left| \dfrac{2}{n-1} \right| \leq 10^{-5} \implies n-1 \geq 2 \times 10^5 \implies n \geq 200001
\]
故需 \( \boxed{200001} \) 次迭代. 
\end{solution}


% =============== 题6 =============== 
\question{}设 \( A \in \mathbb{C}^{n \times n} \) 有实特征值并满足 \(\lambda_1 > \lambda_2 \geq \cdots \geq \lambda_{n-1} > \lambda_n\). 现应用幂法于矩阵 \(A - \mu I\). 试证: 选择 \(\mu = \frac{1}{2}(\lambda_2 + \lambda_n)\) 时, 所产生的向量序列收敛到属于 \(\lambda_1\) 的特征向量的速度最快. 

\begin{solution}
\textbf{证明:} 由于 \(A\) 特征值满足 \(\lambda_1 > \lambda_2 \geq \cdots \geq \lambda_{n-1} > \lambda_n\), 不论 \(\mu\) 取何值, \(A - \mu I\) 按模最大的特征值为 \(\lambda_1 - \mu\) 或 \(\lambda_n - \mu\), 因本题要求收敛到 \(\lambda_1\), 所以首先选择 \(\mu\) 使得
\[
|\lambda_1 - \mu| > |\lambda_n - \mu|
\]
由此可知
\[
\mu < \frac{1}{2}(\lambda_1 + \lambda_n)
\]
定义
\[
f(\mu) := \frac{\max\limits_{2 \leq i \leq n} |\lambda_i - \mu|}{|\lambda_1 - \mu|}
\]
则 \(f(\mu)\) 表征了最大模特征值与次最大模特征值的分离程度, \(f(\mu)\) 越小, 幂法收敛速度越快, 所以有: 

\begin{enumerate}
    \item
    当 \(|\lambda_2 - \mu| \geq |\lambda_n - \mu|\) , 即 \(\mu \le \frac{1}{2}(\lambda_2 + \lambda_n)\) 时: 
    \[
    f(\mu) = \frac{|\lambda_2 - \mu|}{|\lambda_1 - \mu|} = \frac{\lambda_2 - \mu}{\lambda_1 - \mu} = 1 - \frac{\lambda_1 - \lambda_2}{\lambda_1 - \mu}  
    \]
    函数单调递减, 在\(\mu = \frac{1}{2}(\lambda_2 + \lambda_n)\) 处取最小值
    \item
    当 \(|\lambda_2 - \mu| < |\lambda_n - \mu|\) , 即 \(\mu > \frac{1}{2}(\lambda_2 + \lambda_n)\) 时: 
    \[
    f(\mu) = \frac{|\lambda_n - \mu|}{|\lambda_1 - \mu|} = \frac{\mu - \lambda_n}{\lambda_1 - \mu} = \frac{\lambda_1 - \lambda_n}{\lambda_1 - \mu} - 1
    \]
    函数单调递增, 此时 \(f(\mu) > f(\frac{\lambda_2 + \lambda_n}{2}) \)恒成立
\end{enumerate}
综上, 当 \(\mu = \frac{1}{2}(\lambda_2 + \lambda_n)\) 时, $ f(\mu) $ 为最小值, 此时收敛速度最快, 命题得证. 
\end{solution}


% =============== 题7 ===============   
\question{}
设 \( A \in \mathbb{C}^{n \times n} \), \( x \in \mathbb{C}^n \), \( X = [x, Ax, \cdots, A^{n-1}x] \). 证明: 如果 \( X \) 是非奇异的, 则 \( X^{-1}AX \) 是上 Hessenberg 矩阵. 

\begin{solution}
因为 \( X = [x, Ax, \cdots, A^{n-1}x] \) 非奇异, 所以 \(\{x, Ax, \cdots, A^{n-1}x\}\) 线性无关. 

设 \( H = X^{-1}AX = [h_{ij}] \in \mathbb{C}^{n \times n} \), 则 \( AX = XH \). 
考虑 \( AX = XH \) 的第 \( k \) 列, 由矩阵乘法得: 
\[
A \cdot (A^{k-1}x) = A^k x = X h_{k} = \sum_{i=1}^{n} h_{ik} A^{i-1}x, \quad 1 \leq k \leq n. 
\]

当 \( 1 \leq k \leq n-1 \) 时, 
移项后: 
\[
\sum_{\substack{i=1 \\ i \neq k+1}}^{n} h_{ik} A^{i-1}x + (h_{k+1, k} - 1) A^k x = 0
\]

由于 \(\{x, Ax, \cdots, A^{n-1}x\}\) 线性无关, 各项系数必须为零: 
\[
h_{ik} = 0, 1 \le i \le n, i \neq k+1
\]
\[
h_{k+1, k} = 1, 1 \le k \le n - 1
\]

因此, 
\[
H = \begin{pmatrix}
0 & 0 & 0 & \cdots & h_{1n} \\
1 & 0 & 0 & \cdots & h_{2n} \\
0 & 1 & 0 & \cdots & h_{3n} \\
\vdots & \ddots & \ddots & \ddots & \vdots \\
0 & \cdots & 0 & 1 & h_{nn}
\end{pmatrix}
\]
所以, \( X^{-1}AX = H \) 是上 Hessenberg 矩阵.
\end{solution}


% =============== 题8 ===============  
\question{}
设 \( H \) 是一个不可约的上 Hessenberg 矩阵. 证明: 存在一个对角阵 \( D \), 使得 \( D^{-1}HD \) 的次对角元均为 1, 此时 \( \kappa_2(D) = \|D\|_2\|D^{-1}\|_2 \) 是多少?
\begin{solution}设 \( H \) 的次对角元为 \( a_1, \ldots, a_{n-1} \). 由于 \( H \) 是不可约的上 Hessenberg 矩阵, 所以 \( a_i \neq 0 \), \( \forall 1 \le i \le n-1 \). 
取对角矩阵 \( D = \diag(d_1, d_2, \ldots, d_n) \), \( d_i \neq 0 \), \( \forall 1 \le i \le n \), 则
\[
D^{-1} = \diag\left( \frac{1}{d_1}, \frac{1}{d_2}, \ldots, \frac{1}{d_n} \right) 
\]
下考虑在矩阵 \( D^{-1}HD \) 中的元素:  \\ 
在本题中, 由于左乘对角阵 $ D^{-1} $ 相当于让矩阵 $ H $ 的对应行向量分别乘 $ \dfrac{1}{d_i} $ , 右乘对角阵 $ D $ 相当于让矩阵 $ H $ 的对应列向量分别乘 $ d_j $ , 
于是有: 
\[
(D^{-1}HD)_{ij} = \frac{d_j}{d_i} h_{ij}
\]
要求次对角元(即 \( i = j+1 \) 的元素)为 1, 即\( \forall 1 \leq j \leq n-1 \)
\[
\frac{d_j}{d_{j+1}} a_j = 1 \quad \implies \quad d_{j+1} = a_j d_j \quad (1 \leq j \leq n-1)
\]

由此得到一个欠定方程组: 
\[
Ad = 0, A \in \mathbb{C}^{(n-1) \times n}, d \in \mathbb{C}^n, d = (d_1, \dots, d_n)^{\T}
\]
\[
A = 
\begin{bmatrix}
a_1 & -1 & & \\
    & a_2 & -1 & \\
    &     & \ddots & \ddots \\
    &     &        & a_{n-1} & -1
\end{bmatrix}
\]
对这个方程, 给定 \(d_1 \neq 0\) 后, 解得
\[
d_2 = a_1 d_1, \quad d_3 = a_1 a_2 d_1, \quad \cdots, \quad d_n = a_1 \cdots a_{n-1} d_1
\]
所以得到所求对角阵
\[
D = \operatorname{diag}\left( d_1, \  a_1 d_1, \  \cdots, \  \prod_{i=1}^{n-1} a_i d_1 \right)
\]
故所求矩阵$D$一定存在, 再求条件数: 

由矩阵 2-范数的性质知, 对于任一矩阵 \( A \in \mathbb{C}^{n \times n} \)
\[
||A||_2 = \sqrt{\rho(A^{\T}A)}
\]
所以有
\[
||D||_2 = \sqrt{\rho(D^{\T}D)} = \sqrt{\rho(D^2)} = \max_{1 \leq i \leq n} |d_i|
\]
\[
||D^{-1}||_2 = \sqrt{\rho((D^{-1})^{\T}D^{-1})} = \sqrt{\rho((D^{-1})^2)} = \max_{1 \leq j \leq n} |d_j^{-1}| = \frac{1}{\min_{1 \leq j \leq n} |d_j|}
\]
所以
\[
\kappa_2(D) = ||D||_2||D^{-1}||_2 = \frac{\max_{1 \leq i \leq n} |d_i|}{\min_{1 \leq j \leq n} |d_j|}
= \frac{\max_{1 \leq i \leq n} \prod_{k=1}^{i} |a_{k-1}|}{\min_{1 \leq j \leq n} \prod_{l=1}^{j} |a_{l-1}|} 
\]
其中, \( a_0 = 1 \).
\end{solution}


% =============== 题9 ===============
\question{}
设 \( H \) 是一个奇异不可约上 Hessenberg 矩阵. 证明: 进行一次基本的 QR 迭代后, \( H \) 的零特征值将出现. 

\begin{solution}由奇异性和不可约性知
\( \operatorname{rank}(H) = n-1\), 且其下次对角线元素全非零. 


对 \( H \) 进行QR分解. 由于 \( H \) 是上Hessenberg矩阵, 可通过 \( n-1 \) 次Givens变换实现: 
\[
H = QR, \quad Q \in \mathbb{R}^{n \times n} \text{ 正交}, \quad R \in \mathbb{R}^{n \times n} \text{ 上三角}. 
\]
因 \( \operatorname{rank}(H) = n-1 \) 且Givens变换保秩, 所以\( \operatorname{rank}(R) = n-1 \). 结合 \( H \) 不可约的性质: 
\begin{itemize}
    \item \( R \) 的前 \( n-1 \) 个对角元非零(不可约性保证在Givens变换过程中变化后的4个元素里左上角元素一定非0). 
    \item \( R \) 的最后一行全零(因为是秩为 \( n-1 \)的上三角矩阵). 
\end{itemize}
故 \( R \) 可写为分块形式: 
\[
R = \begin{pmatrix} R_1 \\ 0 \end{pmatrix}, \quad R_1 \in \mathbb{R}^{(n-1) \times n} \text{ 满秩上三角}
\]

QR迭代后的矩阵为: 
\[
\tilde{H} = RQ = \begin{pmatrix} R_1 \\ 0 \end{pmatrix} Q = \begin{pmatrix} R_1 Q \\ 0 \end{pmatrix}
\]
上式表明: \( \tilde{H} \) 的最后一行全零, 自然有零特征值出现, 命题得证. 

\end{solution}



\end{questions}

\end{document}