\documentclass[12pt, answers]{exam}     % 试卷
\usepackage{ctex}          % 中文支持
\usepackage{amsmath, amssymb, amsthm}       % 数学公式包
\usepackage{enumitem}
\usepackage{geometry}      % 设置页边距   
\usepackage{float}     
\usepackage[ruled, noline]{algorithm2e}  % 伪代码      
\geometry{a4paper,margin = 1.5cm}

% 定义环境
\SetKwProg{Func}{function:}{}{end}   % 伪代码样式
\newcommand{\anothersolution}{\par\noindent\textbf{另解:}}

% 自定义命令
\newcommand{\rank}{\operatorname{rank}}
\newcommand{\tr}{\operatorname{tr}}
\newcommand{\diag}{\operatorname{diag}}
\newcommand{\XLS}{\mathcal{X}_{\mathrm{LS}}}
\newcommand{\R}{\mathbb{R}}
\newcommand{\C}{\mathbb{C}}
\newcommand{\A}{\mathcal{A}}
\newcommand{\range}{\mathcal{R}}
\newcommand{\argmin}{\operatorname*{argmin}} 
\newcommand{\T}{\mathrm{T}}
\newcommand{\HH}{\mathrm{H}}

% 标题部分  
\title{数值线代の题库一}
\author{by 23大数据miracle}
\date{}

\begin{document}

\maketitle

\begin{questions}
% =============== 题1 ===============
\question{}
写出改进平方根法的算法, 并分析其复杂度. 

\begin{solution}
改进平方根法: 对对称正定阵进行改进平方根分解 $A = LDL^{\T}$, 其中$L$是对角线上都为1的下三角矩阵, $D$是对角矩阵.

\begin{algorithm}[H]
\SetKwFunction{Fldltsolve}{[L,D]=ldltSolve}
\Func{\Fldltsolve{$A$}}{
    \For{$j = 1: n$}{
        \For{$i = 1: j-1$}{
            $v(i) = A(j, i)A(i, i)$ \\
        }
        $A(j, j) = A(j, j) - A(j, 1: j - 1)v(1: j - 1)$ \\
        $A(j + 1: n, j) = \left(A(j + 1: n, j) - A(j + 1: n, 1: j - 1)v(1: j - 1)\right)/A(j, j)$
    }
}
\caption{用改进平方根法求$L$和$D$, 使得$A = LDL^{\T}$. 修改后的$A$对角线上元素构成$D$, $A$左下角非对角线元素是$L$左下角非对角线元素.}
\end{algorithm}

\textbf{复杂度分析:}
\begin{enumerate}
    \item 对于每个固定的$j$ ($1 \leq j \leq n$):
        \begin{itemize}
            \item 计算$v(i)$: 需要$j-1$次乘法运算
            \item 计算$D(j,j)$: 需要$2(j-1)$次运算
            \item 计算$L(i,j)$: 需要$(n-j)[2(j-1)+1] = (n-j)(2j-1)$次运算
        \end{itemize}
    \item 因此对于每个$j$, 运算次数为: $3(j-1) + (n-j)(2j-1)$
    \item 对$j$从$1$到$n$求和, 总运算次数为: 
        \begin{align*}
            &\hspace{5mm} \sum_{j=1}^n \left[3(j-1) + (n-j)(2j-1)\right] \\
            &= -2\sum_{j=1}^n \left[j^2 + (4 + 2n)j - (n + 3)\right] \\
            &= -2 \cdot \frac{n(n+1)(2n+1)}{6} + (4+2n) \cdot \frac{n(n+1)}{2} - n(n+3) \\
            &= \frac{1}{3}n^3 + n^2 - \frac{4}{3}n
        \end{align*}
        所以总运算次数为$\frac{1}{3}n^3 + O(n^2)$, 当$n$很大时, 时间复杂度为$O(n^3)$. 
\end{enumerate}
\end{solution}

% =============== 题2 ===============
\question{}
写出计算列主元的三角分解的算法, 并分析其复杂度. 

\begin{solution}
列主元的三角分解: 分解 $PA = LU$, 其中$P$是置换矩阵, $L$是单位下三角矩阵, $U$是上三角矩阵. 

\begin{algorithm}[H]
\SetKwFunction{FLU}{[L,U,P]=lup}
\Func{\FLU{$A$}}{
    $P = I_n$ \\
    \For{$k = 1: n-1$}{
        确定 $p \in \{k, k+1, \ldots, n\}$, 使得\\ 
        \hspace{1cm}$|A(p, k)| = \max \{|A(i, k)| : i = k, k+1, \ldots, n \}$ \\
        $A(k,1:n) \leftrightarrow A(p,1:n)$ (交换第$k$行和第$p$行)\\
        $P(k,1:n) \leftrightarrow P(p,1:n)$ (记录置换矩阵$P$)\\
        \eIf{$A(k, k) \neq 0$}{
            $A(k+1:n,k) = A(k+1:n,k)/A(k, k)$ \\
            $A(k+1:n,k+1:n) = A(k+1:n,k+1:n) - A(k+1:n,k) A(k,k+1:n)$
        }{
            stop \ (矩阵奇异)
        }
    }
}
\caption{用列主元三角分解求$L$和$U$以及$P$, 使得$PA = LU$. 修改后的$U$上三角部分(含对角线)储存$U$, $L$下三角部分(不含对角线)储存$L$.}
\end{algorithm}

\textbf{复杂度分析: }
\begin{enumerate}
    \item 对于每个固定的$k$ ($1 \leq k \leq n-1$):
        \begin{itemize}
            \item 计算乘数: 需要$n-k$次运算
            \item 更新矩阵: 需要$2(n-k)^2$次运算
        \end{itemize}
    \item 因此对于每个$k$, 运算次数为: $(n-k) + 2(n-k)^2$
    \item 对$k$从$1$到$n-1$求和, 总运算次数为: 
        \begin{align*}
        &\hspace{5mm} \sum_{k=1}^{n-1} [(n-k) + 2(n-k)^2] \\
        &= \sum_{m=1}^{n-1} m + 2\sum_{m=1}^{n-1} m^2 \quad (\text{令} m = n-k) \\
        &= \frac{(n-1)n}{2} + 2 \cdot \frac{(n-1)n(2n-1)}{6} \\
        &= \frac{2}{3}n^3 - \frac{1}{2}n^2 - \frac{1}{6}n
        \end{align*}
        所以总运算次数为$\frac{2}{3}n^3 + O(n^2)$, 当$n$很大时, 时间复杂度为$O(n^3)$. 
\end{enumerate}
\end{solution}

% =============== 题3 ===============
\question{}
求 Gauss 变换$L_{19} = I - l_1e_1^{\T} + l_9e_9^{\T} $的逆. 

\begin{solution}
注意到有分解
\[
L_{19} = (I - l_1 e_1^{\T})(I + l_9 e_9^{\T})
\]
又
\begin{align*}
(I - l_1 e_1^{\T})^{-1} &= I + l_1 e_1^{\T} \\
(I + l_9 e_9^{\T})^{-1} &= I - l_9 e_9^{\T}
\end{align*}
故
\begin{align*}
L_{19}^{-1} &= \left((I - l_1 e_1^{\T})(I + l_9 e_9^{\T})\right)^{-1} \\
&= (I + l_9 e_9^{\T})^{-1}(I - l_1 e_1^{\T})^{-1} \\
&= (I - l_9 e_9^{\T})(I + l_1 e_1^{\T}) \\
&= I + l_1 e_1^{\T} - l_9 e_9^{\T} - l_{9,1} (l_9 e_1^{\T})
\end{align*}
($l_{9,1}$表示$l_1$的第9个元素).
\anothersolution{}
设$L_{19} \in \R^{n \times n}$, 于是有
\[
L_{19} = \begin{bmatrix}
1 & 0 & \cdots & 0 & 0 & \cdots & 0 \\
-l_{2,1} & 1 & \cdots & 0 & 0 & \cdots & 0 \\
\vdots & \vdots & \ddots & \vdots & \vdots & \ddots & \vdots \\
-l_{9,1} & 0 & \cdots & 1 & 0 & \cdots & 0 \\
-l_{10,1} & 0 & \cdots & l_{10,9} & 1 & \cdots & 0 \\
\vdots & \vdots & \ddots & \vdots & \vdots & \ddots & \vdots \\
-l_{n,1} & 0 & \cdots & l_{n,9} & 0 & \cdots & 1
\end{bmatrix}
\]
做初等行变换, 即从
\[
[L_{19} \mid I] = 
\left[
\begin{array}{ccccccc|ccccccc}
1 & 0 & \cdots & 0 & 0 & \cdots & 0 & 1 & 0 & \cdots & 0 & 0 & \cdots & 0 \\
-l_{2,1} & 1 & \cdots & 0 & 0 & \cdots & 0 & 0 & 1 & \cdots & 0 & 0 & \cdots & 0 \\
\vdots & \vdots & \ddots & \vdots & \vdots & \ddots & \vdots & \vdots & \vdots & \ddots & \vdots & \vdots & \ddots & \vdots \\
-l_{9,1} & 0 & \cdots & 1 & 0 & \cdots & 0 & 0 & 0 & \cdots & 1 & 0 & \cdots & 0 \\
-l_{10,1} & 0 & \cdots & l_{10,9} & 1 & \cdots & 0 & 0 & 0 & \cdots & 0 & 1 & \cdots & 0 \\
\vdots & \vdots & \ddots & \vdots & \vdots & \ddots & \vdots & \vdots & \vdots & \ddots & \vdots & \vdots & \ddots & \vdots \\
-l_{n,1} & 0 & \cdots & l_{n,9} & 0 & \cdots & 1 & 0 & 0 & \cdots & 0 & 0 & \cdots & 1 \\
\end{array}
\right]
\]
有
\[
\left[
\begin{array}{ccccccc|ccccccc}
1 & 0 & \cdots & 0 & 0 & \cdots & 0 & 1 & 0 & \cdots & 0 & 0 & \cdots & 0 \\
0 & 1 & \cdots & 0 & 0 & \cdots & 0 & l_{2,1} & 1 & \cdots & 0 & 0 & \cdots & 0 \\
\vdots & \vdots & \ddots & \vdots & \vdots & \ddots & \vdots & \vdots & \vdots & \ddots & \vdots & \vdots & \ddots & \vdots \\
0 & 0 & \cdots & 1 & 0 & \cdots & 0 & l_{9,1} & 0 & \cdots & 1 & 0 & \cdots & 0 \\
0 & 0 & \cdots & l_{10,9} & 1 & \cdots & 0 & l_{10,1} & 0 & \cdots & 0 & 1 & \cdots & 0 \\
\vdots & \vdots & \ddots & \vdots & \vdots & \ddots & \vdots & \vdots & \vdots & \ddots & \vdots & \vdots & \ddots & \vdots \\
0 & 0 & \cdots & l_{n,9} & 0 & \cdots & 1 & l_{n,1} & 0 & \cdots & 0 & 0 & \cdots & 1 \\
\end{array}
\right]
\]
即
\[
L_{19}^{-1} = 
\left[
\begin{array}{ccccccc}
1 & 0 & \cdots & 0 & 0 & \cdots & 0 \\
l_{2,1} & 1 & \cdots & 0 & 0 & \cdots & 0 \\
\vdots & \vdots & \ddots & \vdots & \vdots & \ddots & \vdots \\
l_{9,1} & 0 & \cdots & 1 & 0 & \cdots & 0 \\
l_{10,1} - l_{9,1} l_{10,9} & 0 & \cdots & -l_{10,9} & 1 & \cdots & 0 \\
\vdots & \vdots & \ddots & \vdots & \vdots & \ddots & \vdots \\
l_{n,1} - l_{9,1} l_{n,9} & 0 & \cdots & -l_{n,9} & 0 & \cdots & 1 \\
\end{array}
\right]
\]
故$ L_{19}^{-1} = I + l_1 e_1^{\T} - l_9 e_9^{\T} - l_{9,1} (l_9 e_1^{\T})$.
\end{solution}

% =============== 题4 ===============
\question{}
设 \(S, T \in \R^{n \times n}\) 为两个上三角阵, 且线性方程组 \((ST - \lambda I)x = b\) 是非奇异的. 给出一种算法(要求运算量为\(O(n^2)\))来求解此方程组. 
\begin{solution}
算法核心思路: 
\\通过引入辅助向量 \(y = Tx\), 将原方程 \((ST - \lambda I)x = b\) 转化为\(y = Tx\)和\(S y - \lambda x = b\)两个方程. 利用 \(S\) 和 \(T\) 的上三角结构, 从最后一行开始向前求解. 
\\对于每一行 \(i\), 利用已经求解的 \(x(j)\) 和 \(y(j)\)(\(j > i\))计算当前行的 \(x(i)\) 和 \(y(i)\), 避免显式计算 \(ST\), 从而将复杂度降低到 \(O(n^2)\). 

具体分析:

对于第 \(i\) 行(从 \(n\) 到 \(1\)), 有
\[
y_i = T_{ii} x_i + \sum_{j=i+1}^{n} T_{ij} x_j
\]
以及
\[
S_{ii} y_i + \sum_{j=i+1}^{n} S_{ij} y_j - \lambda x_i = b_i
\]
将 \(y_i\) 的表达式代入第二个方程, 得到
\[
S_{ii} \left( T_{ii} x_i + \sum_{j=i+1}^{n} T_{ij} x_j \right) + \sum_{j=i+1}^{n} S_{ij} y_j - \lambda x_i = b_i
\]
整理后: 
\[
(S_{ii} T_{ii} - \lambda) x_i + S_{ii} \sum_{j=i+1}^{n} T_{ij} x_j + \sum_{j=i+1}^{n} S_{ij} y_j = b_i
\]
因此, \(x_i\) 可表示为
\[
x_i = \frac{ b_i - S_{ii} \sum_{j=i+1}^{n} T_{ij} x_j - \sum_{j=i+1}^{n} S_{ij} y_j }{ S_{ii} T_{ii} - \lambda }
\]
其中 \(S_{ii} T_{ii} - \lambda \neq 0 \)(方程组非奇异). 而一旦求得 \(x_i\), 即可继续计算 \(y_i\). 写成算法如下:

\begin{algorithm}[H]
\SetKwFunction{FSolve}{[x]=solve}
\Func{\FSolve{$S$, $T$, $\lambda$, $b$}}{
    \For{\(i = n: 1\)}{
        \(\text{sum}_1 = 0\) \\
        \For{\(j = i+1: n\)}{
            \(\text{sum}_1 = \text{sum}_1 + T(i,j) x(j)\)
        }
        \(\text{sum}_2 = 0\) \\
        \For{\(j = i+1: n\)}{
            \(\text{sum}_2 = \text{sum}_2 + S(i,j) y(j)\)
        }
        \(x(i) = (b(i) - S(i,i) \cdot \text{sum}_1 - \text{sum}_2) / (S(i,i) T(i,i) - \lambda\))  (分母一定非0)\\
        \(y(i) = T(i,i) x(i) + \text{sum}_1\) \\
    }
}
\caption{求解上三角矩阵方程组 \((ST - \lambda I)x = b\) 的 \(O(n^2)\) 算法.}
\end{algorithm}
\end{solution}

% =============== 题5 ===============
\question{}证明: 若$ A \in \R^{n \times n} $有三角分解且非奇异, 则L和U是唯一的.

\begin{solution}假设非奇异矩阵$A$有两个不同的LU分解: 
\[
A = L_1U_1, \quad A = L_2U_2
\]
其中$L_1,L_2$是单位下三角矩阵(一定非奇异), $U_1,U_2$是上三角矩阵. 

因为$A$非奇异, 所以$U_1$和$U_2$也都是非奇异的. 于是由 
\[ L_1U_1 = L_2U_2 \]
得
\[ L_2^{-1}L_1 = U_2U_1^{-1} \]
注意到等式左侧一定是下三角矩阵, 等式右侧一定是上三角矩阵. 

因此$L_2^{-1}L_1 = U_2U_1^{-1}$必须是对角阵. 又因为$L_2^{-1}L_1$是单位下三角矩阵, 所以只能是单位矩阵, 即有
\[ L_2^{-1}L_1 = I \Rightarrow L_1 = L_2 \]
\[ U_2U_1^{-1} = I \Rightarrow U_1 = U_2 \]
与假设矛盾, 故分解唯一.
\end{solution}


% =============== 题6 ===============
\question{}
设 \( A = [a_{ij}] \in \R^{n \times n} \) 的定义如下: 
\[ a_{ij} =
\begin{cases} 
1, & \text{如果 } i = j \text{ 或 } j = n \\
-1, & \text{如果 } i > j \\
0, & \text{其他}
\end{cases}
\]

证明: \( A \) 有满足 \( |l_{ij}| \leq 1 \) 和 \( u_{nn} = 2^{n-1} \) 的三角分解. 

\begin{solution}
由题意
\[ A = \begin{pmatrix}
1 & 0 & \cdots & 0 & 1 \\
-1 & 1 & \cdots & 0 & 1 \\
-1 & -1 & \cdots & 0 & 1 \\
\vdots & \vdots & \ddots & \vdots & \vdots \\
-1 & -1 & \cdots & 1 & 1 \\
\end{pmatrix} \]
则直接 LU 分解有\( A = LU \), 其中
\[ L = \begin{pmatrix}
1 & 0 & \cdots & 0 & 0 \\
-1 & 1 & \cdots & 0 & 0 \\
-1 & -1 & \cdots & 0 & 0 \\
\vdots & \vdots & \ddots & \vdots & \vdots \\
-1 & -1 & \cdots & -1 & 1 \\
\end{pmatrix}, \quad
U = \begin{pmatrix}
1 & 0 & \cdots & 0 & 1 \\
0 & 1 & \cdots & 0 & 2 \\
0 & 0 & \cdots & 0 & 4 \\
\vdots & \vdots & \ddots & \vdots & \vdots \\
0 & 0 & \cdots & 1 & 2^{n-1} \\
\end{pmatrix} \]
直接验证满足题意, 证毕. 
\end{solution}

% =============== 题7 ===============
\question{}
设 \( A \) 对称且 \( a_{11} \neq 0 \), 并假定经过一步Gauss消去后, \( A \) 具有如下形状: 
\[
\begin{pmatrix}
a_{11} & a_1^{\T} \\
0 & A_2
\end{pmatrix}
\]
证明: \( A_2 \) 仍是对称阵. 

\begin{solution}
将对称矩阵 \( A \) 分块表示为: 
\[
A = 
\begin{pmatrix}
a_{11} & b_1^{\T} \\
b_1 & A_{22}^{(0)}
\end{pmatrix}
\]
其中 \( b_1 = (a_{21}, \ldots, a_{n1})^{\T} \), 且 \( A_{22}^{(0)} \) 是对称矩阵. 

进行一步Gauss消去时对应的变换矩阵为: 
\[
L_1 = 
\begin{pmatrix}
1 & 0 \\
-\frac{b_1}{a_{11}} & I
\end{pmatrix}
\]

则 
\[
L_1 A = 
\begin{pmatrix}
1 & 0 \\
-\frac{b_1}{a_{11}} & I
\end{pmatrix}
\begin{pmatrix}
a_{11} & b_1^{\T} \\
b_1 & A_{22}^{(0)}
\end{pmatrix}
=
\begin{pmatrix}
a_{11} & b_1^{\T} \\
0 & A_{22}^{(0)} - \frac{b_1 b_1^{\T}}{a_{11}}
\end{pmatrix}
\]

根据题意, 比较可得: 
\[
a_1 = b_1, \quad A_2 = A_{22}^{(0)} - \frac{b_1 b_1^{\T}}{a_{11}}
\]

由于 \( A_{22}^{(0)} \) 和 \( b_1 b_1^{\T} \) 对称, 
故 \( A_2 \) 是对称矩阵.
\end{solution}

% =============== 题8 ===============
\question{}设 \( A = (a_{ij}) \in \R^{n \times n} \) 是严格对角占优, 即$A$满足
\[ |a_{kk}| > \sum_{\substack{j=1 \\ j \neq k}}^n |a_{kj}|, \quad k = 1,\ldots,n \]
又设经过一次Gauss消去后, \( A \) 具有如下形状: 
\[
\begin{pmatrix}
a_{11} & a_1^{\T} \\
0 & A_2
\end{pmatrix}
\]
证明: 矩阵 \( A_2 \) 仍是严格对角占优阵.

\begin{solution}
类似题7可知: 
\[ A_2 = A_{22}^{(0)} - \frac{\alpha_1 a_1^{\T}}{a_{11}} \]
其中 \( \alpha_1 = (a_{21}, \ldots, a_{n1})^{\T} \), \( a_1^{\T} = (a_{12}, \ldots, a_{1n}) \), \( A_{22}^{(0)} \) 是 \( A \) 的右下$(n - 1) \times (n - 1)$阶子矩阵. 

设 \( A_2 = (b_{ij})_{i,j=2}^n \), 即有 
\[ b_{ij} = a_{ij} - \frac{a_{i1}a_{1j}}{a_{11}} \]

考虑对角元有
\[ |b_{kk}| = \left| a_{kk} - \frac{a_{k1}a_{1k}}{a_{11}} \right| \geq |a_{kk}| - \left| \frac{a_{k1}}{a_{11}} \right| | a_{1k} |\]

同行的非对角元和: 
\[ \sum_{\substack{j=2 \\ j \neq k}}^n |b_{kj}| = \sum_{\substack{j=2 \\ j \neq k}}^n \left| a_{kj} - \frac{a_{k1}a_{1j}}{a_{11}} \right| \leq \sum_{\substack{j=2 \\ j \neq k}}^n |a_{kj}| + \left| \frac{a_{k1}}{a_{11}} \right| \sum_{\substack{j=2 \\ j \neq k}}^n |a_{1j}| \]

由 \( A \) 严格对角占优有
\[ |a_{11}| > \sum_{j=2}^n |a_{1j}| = |a_{1k}| + \sum_{\substack{j=2 \\ j \neq k}}^n |a_{1j}| \]
\[ |a_{kk}| > |a_{k1}| + \sum_{\substack{j=2 \\ j \neq k}}^n |a_{kj}| ,\quad \forall k \ne 1 \]

代入前式得: \( \forall k = 2, 3, \ldots, n \)有
\begin{align*}
|b_{kk}| &\geq |a_{kk}| - \left| \frac{a_{k1}}{a_{11}} \right| | a_{1k} | \\
&> (|a_{k1}| + \sum_{\substack{j=2 \\ j \neq k}}^n |a_{kj}|) + \left| \frac{a_{k1}}{a_{11}} \right| ( \sum_{\substack{j=2 \\ j \neq k}}^n |a_{1j}| - |a_{11}| )  \\
&= \sum_{\substack{j=2 \\ j \neq k}}^n |a_{kj}| + \left| \frac{a_{k1}}{a_{11}} \right| \sum_{\substack{j=2 \\ j \neq k}}^n |a_{1j}|  \\
&\geq \sum_{\substack{j=2 \\ j \neq k}}^n |b_{kj}|
\end{align*}
故 \( A_2 \) 严格对角占优. 
\end{solution}

% =============== 题9 ===============
\question{}
设 \( A \) 为正定阵. 若对 \( A \) 执行Gauss消去法一步后产生一个形为
\[
\begin{pmatrix}
a_{11} & a_1^{\T} \\
0 & A_2
\end{pmatrix}
\]
的矩阵. 证明: \( A_2 \) 仍是正定阵. 
\begin{solution}
由题7可知: 
\[ A_2 = A_{22}^{(0)} - \frac{a_1 a_1^{\T}}{a_{11}} \]
其中 \( a_1 = (a_{21}, \ldots, a_{n1})^{\T} \), \( A_{22}^{(0)} \) 是 \( A \) 的右下$(n - 1) \times (n - 1)$阶子矩阵.

同样设
\[
L_1 = 
\begin{pmatrix}
1 & 0 \\
-\frac{a_1}{a_{11}} & I
\end{pmatrix}
\]

考虑合同变换后的矩阵
\[
L_1 A L_1^{\T} = 
\begin{pmatrix}
a_{11} & 0 \\
0 & A_2
\end{pmatrix}
\]
同样正定.

对任意非零列向量 \( x \in \R^{n-1} \), 令 \( y = L_1^{\T} \begin{pmatrix} 0 \\ x \end{pmatrix} \ne 0 \), 则: 
\[
y^{\T} A y = \begin{pmatrix} 0 \\ x \end{pmatrix}^{\T} L_1 A L_1^{\T} \begin{pmatrix} 0 \\ x \end{pmatrix} = x^{\T} A_2 x
\]

由于 \( A \) 正定, 有\( y^{\T} A y > 0 \), 故
\[
x^{\T} A_2 x > 0 \quad \forall x \neq 0
\]

因此 \( A_2 \) 正定.
\end{solution}

% =============== 题10 ===============
\question{}证明: 如果 \( A^{\T} \in \R^{n \times n} \) 为严格对角占优阵. 那么\( A \)有三角分解 \( A = LU \) 且 \( |l_{ij}| < 1 \).
\begin{solution}用归纳假设. 当 \( n = 1 \) 时, 显然符合题意. 假设对于矩阵阶数为 \( (n - 1) \times (n - 1) \) 时结论皆成立, 下归纳证明矩阵阶数为 \( n \times n \) 的情况: \\
由 \( A^{\T} \) 严格对角占优可知: \( A \) 的对角元绝对值大于所在列其他元素绝对值之和, 即有
\[ |a_{11}| > \sum_{i=2}^n |a_{i1}| \]
故一定有\( a_{11} \neq 0 \), 则可以进行第一步Gauss运算, 设
\[ l_{i1} = \frac{a_{i1}}{a_{11}} \]
则有
\[ |l_{i1}| = \left|\frac{a_{i1}}{a_{11}}\right| < 1, \forall i = 2, \ldots, n \]

同时记 \( a_1^{\T} = (a_{12}, \ldots, a_{1n}) \)以及\(l_1 = (l_{21}, \ldots, l_{n1})^{\T} \), 则此时
\[ 
A = \begin{pmatrix}
1 & 0 \\
l_1 & I_{n-1}
\end{pmatrix}
\begin{pmatrix}
a_{11} & a_1^{\T} \\
0 & A_2
\end{pmatrix}
\]
类似题8可知 \( A_2^{\T} \) 严格对角占优. 于是根据归纳假设 \( A_2 \)有三角分解\( A_2 = L_2 U_2 \). \\
记\( L_2 = (l_{ij})_{i,j=2}^n \), 则\( |l_{ij}| < 1, \forall j = 2, \ldots, n, \forall i = j + 1, \ldots, n \). \\
则
\begin{align*}
A &= \begin{pmatrix}
1 & 0 \\
l_1 & I_{n-1}
\end{pmatrix}
\begin{pmatrix}
a_{11} & a_1^{\T} \\
0 & L_2 U_2
\end{pmatrix}  \\
&= \begin{pmatrix}
1 & 0 \\
l_1 & I_{n-1}
\end{pmatrix}
\begin{pmatrix}
1 & 0 \\
0 & L_2
\end{pmatrix}
\begin{pmatrix}
a_{11} & a_1^{\T} \\
0 & U_2
\end{pmatrix}  \\
&= \begin{pmatrix}
1 & 0 \\
l_1 & L_2
\end{pmatrix}
\begin{pmatrix}
a_{11} & a_1^{\T} \\
0 & U_2
\end{pmatrix}  \\
&\triangleq LU 
\end{align*}
此时 \( L = (l_{ij})_{i,j=1}^n \)且\( |l_{ij}| < 1, \forall i > j \). 
证毕.
\end{solution}

% =============== 题11 ===============
\question{}
形如 \( N(y, k) = I - y e_k^{\T} \) 的矩阵称为 Gauss-Jordan 变换, 其中 \( y \in \R^n \). 
\begin{enumerate}[label=(\arabic*)]
    \item 假定 \( N(y, k) \) 非奇异, 求其逆.
    \item 向量 \( x \in \R^n \) 满足何种条件才能保证存在 \( y \in \R^n \), 使得 \( N(y, k)x = e_k \)?
    \item 给出一种利用 Gauss-Jordan 变换求 \( A \in \R^{n \times n} \) 的逆矩阵 \( A^{-1} \) 的算法, 并说明 \( A \) 满足何种条件才能保证算法进行到底. 
\end{enumerate}

\begin{solution}
\begin{enumerate}[label=(\arabic*)]
    \item 首先, 矩阵 \( N(y, k) \) 具有如下结构: 
    \[
    N(y, k) = 
    \begin{bmatrix}
    1 & & & -y_1 & & \\
      & \ddots & & \vdots & & \\
      & & 1 & -y_{k-1} & & \\
      & & & 1 - y_k & & \\
      & & & -y_{k+1} & 1 & \\
      & & & \vdots & & \ddots \\
      & & & -y_n & & & 1
    \end{bmatrix}
    \]
    观察到由初等变换作用可化为只含原对角线元素的对角阵, 从而\( \det(N) = 1 - y_k \). 因为$N(y,k)$非奇异, 故\( 1 - y_k \neq 0 \).
    从而由初等变换法可得 \( N(y, k) \) 的逆为:
    \[
    N(y, k)^{-1} = \begin{bmatrix}
    1 & & & \frac{y_1}{1 - y_k} & & \\
      & \ddots & & \vdots & & \\
      & & 1 & \frac{y_{k-1}}{1 - y_k} & & \\
      & & & \frac{1}{1 - y_k} & & \\
      & & & \frac{y_{k+1}}{1 - y_k} & 1 & \\
      & & & \vdots & & \ddots \\
      & & & \frac{y_n}{1 - y_k} & & & 1
    \end{bmatrix}
    \]
    \item 由 \( N(y, k)x = (I - y e_k^{\T})x = e_k \) 可得
    \[
    x - x_k y = e_k
    \]
    从而
    \[
    y = \frac{x - e_k}{x_k} = \left[ \frac{x_1}{x_k}, \ldots, \frac{x_{k-1}}{x_k}, \frac{x_k - 1}{x_k}, \frac{x_{k+1}}{x_k}, \ldots, \frac{x_n}{x_k} \right]^{\T}
    \]
    为保证 \( y \) 存在, 则需 \( x_k \neq 0 \) 即可.
    \item 设
    \[
    A = 
    \begin{bmatrix}
    a_{11} & \cdots & a_{1n} \\
    \vdots & \ddots & \vdots \\
    a_{n1} & \cdots & a_{nn}
    \end{bmatrix}
    \]
    取
    \[
    y_1 = \left[ 1 - \frac{1}{a_{11}}, \frac{a_{21}}{a_{11}}, \ldots, \frac{a_{n1}}{a_{11}} \right]^{\T}
    \]
    构造 \( N_1 = N(y_1, 1) \)(要求 \( a_{11} \neq 0 \)). 左乘 \( N_1 \) 得
    \[
    A^{(1)} = N_1 A = 
    \begin{bmatrix}
    1 & a_{12}^{(1)} & \cdots & a_{1n}^{(1)} \\
    0 & a_{22}^{(1)} & \cdots & a_{2n}^{(1)} \\
    \vdots & \vdots & \ddots & \vdots \\
    0 & a_{n2}^{(1)} & \cdots & a_{nn}^{(1)}
    \end{bmatrix}
    \]
    一般地, 在第 \( k \) 步, 若 \( a_{kk}^{(k-1)} \neq 0 \), 则取
    \[
    y_k = \left[ \frac{a_{1k}^{(k-1)}}{a_{kk}^{(k-1)}}, \ldots, \frac{a_{k-1,k}^{(k-1)}}{a_{kk}^{(k-1)}}, 1 - \frac{1}{a_{kk}^{(k-1)}}, \frac{a_{k+1,k}^{(k-1)}}{a_{kk}^{(k-1)}}, \ldots, \frac{a_{nk}^{(k-1)}}{a_{kk}^{(k-1)}} \right]^{\T}
    \]
    构造 \( N_k = N(y_k, k) \), 并计算 \( A^{(k)} = N_k A^{(k-1)} \). 经过 \( n \) 步后, 有: 
    \[
    N_n N_{n-1} \cdots N_1 A = E \Rightarrow
    A^{-1} = N_n N_{n-1} \cdots N_1
    \]
    要保证算法进行到底, 需每一步的 \( a_{kk}^{(k-1)} \neq 0 \), 由课本定理可知即 \( A \) 的所有顺序主子阵非奇异. 具体算法见下:
\end{enumerate}
\begin{algorithm}[H]
\SetKwFunction{FGJ}{[B]=GaussJordanInversion}
\Func{\FGJ{$A$}}{
    $B = I_n$ \\
    \For{$k = 1: n$}{
        \If{$A(k, k) = 0$}{
            \textbf{stop}
        }
        \tcp{构造变换向量 $y_k$}
        \For{$i = 1: n$}{
            \If{$i \neq k$}{
                $y(i) = A(i, k) / A(k, k)$ (计算非对角元分量)\\
            }
            \Else{
                $y(i) = 1 - 1 / A(k, k)$ (计算对角元分量)\\
            }
        }
        \tcp{同时对矩阵$A \ \text{和}\ B$左乘 $N_k = N(y_k, k)$}
        \For{$j = 1: n$}{
            $\alpha = A(k, j);\ \beta = B(k, j)$ \\
            \For{$i = 1: n$}{
                $A(i, j) = A(i, j) - y(i) \cdot \alpha$ (更新 $A_{i,j}$)\\
                $B(i, j) = B(i, j) - y(i) \cdot \beta$ (更新 $B_{i,j}$)\\
            }
        }
    }
}
\caption{用 Gauss-Jordan 变换求矩阵 $A$ 的逆矩阵.}
\end{algorithm}
\anothersolution{}
只针对第(1)问: 由 \[(y e_k^{\T})^2 = (y e_k^{\T})(y e_k^{\T}) = y (e_k^{\T} y) e_k^{\T} = y_k (y e_k^{\T})\]
替换$y e_k^{\T} = I - N$有
\[ (I - N)^2 = y_k (I - N) \]
展开后合并同类项有
\[ N^2 + (y_k - 2)N + (1 - y_k)I = 0 \]
又$N$非奇异, 于是
\[ N + (y_k - 2)I + (1 - y_k)N^{-1} = 0 \]
若$y_k = 1$, 则代入$N = I - y e_k^{\T}$可知$N$奇异, 矛盾. 自然有$y_k \neq 1$, 于是
\[ N^{-1} = \frac{(2 - y_k)I - N}{1 - y_k} = \frac{(1 - y_k)I + (I - N)}{1 - y_k} = I + \frac{y e_k^{\T}}{1 - y_k} \]
\end{solution}

% =============== 题12 ===============
\question{}
证明: 如果 \( A \) 是一个带宽为 \( 2m+1 \) 的对称正定带状矩阵. 则其 Cholesky 因子 \( L \) 也是带状矩阵. \( L \) 的带宽是多少?
\begin{solution}
设 \( A \) 是一个 \( n \times n \) 的对称正定带状矩阵, 带宽为 \( 2m + 1 \)(即 \( \forall |i-j| > m \) 时, \( a_{ij} = 0 \)). 
考虑 Cholesky 分解 \( A = LL^{\T} \). 以下证明下三角矩阵 \( L \) 是下带状的, 且带宽为 \( m+1 \)(即当 \( i - k > m \) 时, \( l_{ik} = 0 \)). 

分析$m$有取值范围\( m = 0, 1, \ldots, n - 1 \), 其中当\( m = 0 \)时为对角矩阵, 当\( m = n - 1 \)时为一般矩阵, 且小宽带的矩阵一定是更大宽带的.

于是采用归纳法, 当\( m = n - 1 \)时结论成立, 下证明\( m \geq p \)时成立(即当 \( i - k > p \)时,  \( l_{ik} = 0 \))可以推出\( m = p - 1 \)时结论成立(即当 \( i - k > p - 1 \)时,  \( l_{ik} = 0 \)):

由 Cholesky 分解公式知\( \forall 1 \le i, k \le n \)有
\[
l_{ik} = \frac{1}{l_{kk}} \left( a_{ik} - \sum_{j=1}^{k-1} l_{ij} l_{kj} \right)
\]
考虑当\( i - k = p \)时, 于是 \( i - j > i - k = p \), 有\( l_{ij} = 0 \), 又\( a_{ik} = 0 \), 则求和项 \( l_{ik} = 0 \), 所需得证.

由归纳假设 \( \forall m = 0, 1, \ldots, n - 1 \)都有$L$的带宽为 \( m+1 \), 证毕.
\end{solution}

% =============== 题13 ===============
\question{}若 \( A = LL^{\T} \) 是 \( A \) 的 Cholesky 分解. 证明: \( L \) 的 \( i \) 阶顺序主子阵 \( L_i \) 正好是 \( A \) 的 \( i \) 阶顺序主子阵 \( A_i \) 的 Cholesky 因子. 

\begin{solution}
对矩阵进行分块:
\[
A = \begin{bmatrix}
A_{11} & A_{12} \\
A_{21} & A_{22}
\end{bmatrix}, \quad
L = \begin{bmatrix}
L_{11} & 0 \\
L_{21} & L_{22}
\end{bmatrix}
\]
其中 \( A_{11} \) 和 \( L_{11} \) 是 \( i \times i \) 的矩阵(即 \( i \) 阶顺序主子阵).

根据\( A = LL^{\T} \)展开有 
\[
\begin{bmatrix}
A_{11} & A_{12} \\
A_{21} & A_{22}
\end{bmatrix}
= 
\begin{bmatrix}
L_{11} & 0 \\
L_{21} & L_{22}
\end{bmatrix}
\begin{bmatrix}
L_{11}^{\T} & L_{21}^{\T} \\
0 & L_{22}^{\T}
\end{bmatrix}
= 
\begin{bmatrix}
L_{11}L_{11}^{\T} & L_{11}L_{21}^{\T} \\
L_{21}L_{11}^{\T} & L_{21}L_{21}^{\T} + L_{22}L_{22}^{\T}
\end{bmatrix}
\]

即有
\[
A_{11} = L_{11}L_{11}^{\T}
\]
故 \( L_{11} \) 是 \( A_{11} \) 的 Cholesky 因子.
\end{solution}

% =============== 题14 ===============
\question{}
证明: 设 \( A \in \R^{n \times n} \) 是对称的, 且其前 \( n-1 \) 个顺序主子阵均非奇异. 则 \( A \) 有唯一的分解式
\( A = LDL^{\T} \). 其中 \( L \) 是单位下三角阵, \( D \) 是对角阵. 

\begin{solution}
我们对矩阵的阶数 \( n \) 进行数学归纳法. 
\begin{enumerate}
\item 当$n = 1$时, 取 \( L = [1] \), \( D = [a_{11}] \), 则 \( A = LDL^{\T} \) 成立, 且分解唯一. 
\\ 现假设对于所有 \( k \times k \) 对称矩阵(\( k \leq n-1 \)), 若其前 \( k-1 \) 个顺序主子阵非奇异, 则存在唯一分解 \( A = LDL^{\T} \).
\item 下讨论\( k = n \)时: 
将 \( A \) 分块为
\[
A = \begin{bmatrix}
    A_{n-1} & b^{\T} \\
    b & a_{nn}
\end{bmatrix}
\]
其中 \( A_{n-1} \) 有唯一分解 \( A_{n-1} = L_{n-1}D_{n-1}L_{n-1}^{\T} \). 
又 \( A_{n-1} \) 非奇异, 则 \( L_{n-1} \) 和 \( D_{n-1} \) 非奇异. 设
\[
L = \begin{bmatrix}
    L_{n-1} & 0 \\
    l & 1
\end{bmatrix}, \quad
D = \begin{bmatrix}
    D_{n-1} & 0 \\
    0 & d
\end{bmatrix}
\]
令
\begin{align*}
LDL^{\T} &= \begin{bmatrix}
    L_{n-1} D_{n-1} L_{n-1}^{\T} & L_{n-1} D_{n-1} l^{\T} \\
    l D_{n-1} L_{n-1}^{\T} & l D_{n-1} l^{\T} + d
\end{bmatrix} \\
&= \begin{bmatrix}
    A_{n-1} & b^{\T} \\
    b & a_{nn}
\end{bmatrix} \\
&= A
\end{align*}
于是解得\( l = (L_{n-1}^{\T})^{-1} D_{n-1}^{-1} \), \( d = a_{nn} - l D_{n-1} l^{\T} \) 存在且唯一.

因此 \( L \) 和 \( D \) 存在且唯一, 使得 \( A = LDL^{\T} \), 命题得证. 
\end{enumerate} 
\anothersolution{}
先假设$A$可逆, 从存在性和唯一性两方面来证明:
\begin{enumerate}
    \item \textbf{存在性}: 
    由题目条件可知, 存在单位下三角矩阵 \( L \in \R^{n \times n} \) 和上三角矩阵 \( U \in \R^{n \times n} \) 使得 \( A = LU \), 且 \( U \) 的主对角线元素均不为 0. 令
    \[
    D = \operatorname{diag}(u_{11}, \ldots, u_{nn}), \quad \widetilde{U} = D^{-1}U
    \]
    则$\widetilde{U}$是单位上三角阵. 同时有
    \[
    A = L D \widetilde{U} = A^{\T} = \widetilde{U}^{\T} D L^{\T}    
    \]
    从而
    \[
    D \widetilde{U} L^{-\T} = L^{-1} \widetilde{U}^{\T} D
    \]
    注意到, 左端为上三角阵, 右端为下三角阵, 从而二者为对角阵. 又对角阵 \( D \) 的对角元均不为零, 则$\widetilde{U}L^{-\T}$是对角阵. 又$\widetilde{U}L^{-\T}$对角线上元素都是1, 所以
    \[
    \widetilde{U} L^{-\T} = I
    \]
    故 \( \widetilde{U} = L^{\T} \). 从而 \( A \) 有分解式 \( A = LDL^{\T} \). 

    \item \textbf{唯一性}: 
    若 \( A \) 有分解式
    \[
    A = LDL^{\T} \overset{\triangle}{=} LU
    \]
    则 \( U \) 是上三角矩阵. 由于 \( LU \) 分解的唯一性, 不难得出分解式 \( A = LDL^{\T} \) 的唯一性. 
\end{enumerate}
当$A$不可逆时, 记
\[
A = \begin{bmatrix}
    A_{n-1} & b^{\T} \\
    b & a_{nn}
\end{bmatrix}
\] 
由条件$A_{n-1}$是非奇异的, 则存在唯一的分解 $A_{n-1} = L_{n-1} D_{n-1} L_{n-1}^{\T}$, 其中 $L_{n-1}$ 是单位下三角矩阵, $D_{n-1}$ 是对角矩阵. \\
此时类似于上一个方法通过分块矩阵运算可以证$A$仍然存在唯一的分解式$A=LDL^{\T}$, 自然原命题得证.
\end{solution}

% =============== 题15 ===============
\question{}
设 \( H = A + iB \) 是一个正定 Hermite 矩阵. 其中 \( A, B \in \mathbb{R}^{n \times n} \).
\begin{enumerate}[label=(\arabic*)]
\item 证明: 矩阵
\[
C =
\begin{bmatrix}
A & -B \\
B & A
\end{bmatrix}
\]
是正定对称的.

\item 给出一种仅用实数运算的算法来求解线性方程组
\[
(A + iB)(x + iy) = (b + ic), \quad x, y, b, c \in \mathbb{R}^n
\]
\end{enumerate}

\begin{solution}
\begin{enumerate}[label=(\arabic*)]
\item 由于 \( H \) 是 Hermite 矩阵, 故\( H = H^{\HH} \),即 \[ A + iB = A^{\T} - iB^{\T} \Rightarrow A = A^{\T} \text{且} B = -B^{\T} \]
从而
\[
C = \begin{bmatrix} A & -B \\ B & A \end{bmatrix}
= \begin{bmatrix} A^{\T} & B^{\T} \\ -B^{\T} & A^{\T} \end{bmatrix} 
= C^{\T}
\]
由于 \( H \) 正定, 对于任意非零实向量 \( \begin{bmatrix} u \\ v \end{bmatrix} \in \mathbb{R}^{2n} \), 令 \( z = u + i v \), 则
\[
z^{\HH} H z > 0.
\]
计算可得(中间用到了$A$是对称矩阵以及$B$是反对称矩阵)
\begin{align*}
z^{\HH} H z &= (u^{\T} - i v^{\T}) (A + iB) (u + i v) \\
&= (u^{\T} A + v^{\T} B + iu^{\T} B - iv^{\T} A) (u + i v) \\  
&= u^{\T} A u + v^{\T} B u - u^{\T} B v + v^{\T} A v + iu^{\T} B u - iv^{\T} A u + iu^{\T} A v + iv^{\T} B v\\  
&= u^{\T} A u + v^{\T} A v - 2 u^{\T} B v
\end{align*}
而
\begin{align*}
\begin{bmatrix} u^{\T} & v^{\T} \end{bmatrix} C \begin{bmatrix} u \\ v \end{bmatrix} 
&= \begin{bmatrix} u^{\T} & v^{\T} \end{bmatrix} \begin{bmatrix} A & -B \\ B & A \end{bmatrix} \begin{bmatrix} u \\ v \end{bmatrix} \\
&= \begin{bmatrix} u^{\T} A + v^{\T} B & - u^{\T} B + v^{\T} A \end{bmatrix} \begin{bmatrix} u \\ v \end{bmatrix} \\       
&= u^{\T} A u + v^{\T} B u - u^{\T} B v + v^{\T} A v \\
&= u^{\T} A u + v^{\T} A v - 2 u^{\T} B v
\end{align*}
故
\[
\begin{bmatrix} u^{\T} & v^{\T} \end{bmatrix} C \begin{bmatrix} u \\ v \end{bmatrix} = z^{\HH} H z > 0
\]
综上, 所以$C$是对称正定阵.

\item 由方程 \((A + iB)(x + iy) = b + ic\) 知有
\(
A x - B y = b \text{且} A y + B x = c.
\)
即
\[
\begin{bmatrix} A & -B \\ B & A \end{bmatrix} \begin{bmatrix} x \\ y \end{bmatrix} = \begin{bmatrix} b \\ c \end{bmatrix}
\Longleftrightarrow C \begin{bmatrix} x \\ y \end{bmatrix} = \begin{bmatrix} b \\ c \end{bmatrix}
\]
该问题自然转化为求解由正定矩阵构成的线性方程组.
\end{enumerate}
\begin{algorithm}[H]
\SetKwFunction{FCholesky}{[x,y]=Cholesky\_Solve}
\Func{\FCholesky{$A$, $B$, $b$, $c$}}{
    $C = \begin{bmatrix} A & -B \\ B & A \end{bmatrix};\ d = \begin{bmatrix} b \\ c \end{bmatrix}$ \\
    $\left(L,D\right) = \texttt{ldltSolve}(C)$  (见题1)\\
    \For{$i = 1: 2n$}{
        \For{$j = 1: i - 1$}{
            $d(i) = d(i) - L(i,j)d(j)$ (解方程组 $L d_1 = d$, 用$d$储存新的$d_1$)
        }
    }
    \For{$i = 1: 2n$}{ 
        $d(i) = d(i) / D(i,i)$ (解方程组 $D d_2 = d_1$, 用$d$储存新的$d_2$; 分母一定不为0)\\
    }
    \For{$i = 2n: 1$}{
        \For{$j = i + 1: 2n$}{
            $d(i) = d(i) - L(j,i)d(j)$ (解方程组 $L^{\T} d_3 = d_2$, 用$d$储存最后的$d_3$)
        }
    }
    $x = d(1:n);\ y = d(n+1:2n)$
}
\caption{用Cholesky分解求解复线性方程组 $(A + iB)(x + iy) = (b + ic)$.}
\end{algorithm}
\end{solution}

% =============== 题16 ===============
\question{}
证明: 有限维空间的任意范数都等价.

\begin{solution}
不妨设该有限维空间为$\R^n$, $\| \cdot \|_\alpha$ 和 $\| \cdot \|_\beta$ 是 $\R^n$ 上的任意两个范数. \\
考虑赋范向量空间 $X = (\mathbb{R}^n,\ \| \cdot \|_\alpha)$, 定义函数
\[
f : X \to \mathbb{R}, \quad f(x) = \|x\|_\beta
\]
首先证明 $f$ 连续: 由三角不等式, 对任意 $x, y \in X$, 有
\begin{align*}
\|x\|_{\beta} - \|y\|_{\beta} &= \|x - y + y\|_{\beta} - \|y\|_{\beta} \leq \|x - y\|_{\beta} \\
\|y\|_{\beta} - \|x\|_{\beta} &= \|-y\|_{\beta} - \|-x\|_{\beta} \leq \|x - y\|_{\beta}
\end{align*}
于是
\[
\left| \|x\|_{\beta} - \|y\|_{\beta} \|\right| \leq \|x - y\|_{\beta}       
\]
则对任意 $\varepsilon > 0$, 取 $\delta = \varepsilon$, 则当 $\|x - y\| < \delta$ 时, 
\[
\left| \|x\|_{\beta} - \|y\|_{\beta} \right| \leq \|x - y\|_{\beta} < \varepsilon
\]
故 $f$ 是一致连续的, 自然连续.

由于 $X$ 中的单位球面 $S = \{ x \in \mathbb{R}^n : \|x\|_\alpha = 1 \}$ 紧, 且 $f$ 在 $S$ 上连续, 故存在最小值 $C_1$ 和最大值 $C_2$ 且$0 \leq C_1 \leq C_2$, 使得
\[
C_1 \leq f(x) \leq C_2, \quad \forall x \in S
\]
假设$C_1 = 0$, 即存在$ x_0 \in S $, 使得$\| x_0 \|_{\beta} = 0 \Leftrightarrow x_0 = 0 $. 然而此时就有$\|x_0\|_{\alpha} = 0 \Rightarrow x_0 \notin S$, 矛盾, 故
\[
0 < C_1 \leq f(x) \leq C_2, \quad \forall x \in S
\]
然后, 对任意 $x \in \mathbb{R}^n\ \text{且}\ x \neq 0$, 令
\[
u = \frac{x}{\|x\|_\alpha} \in S
\]
则
\[
0 < C_1 \leq f(u) = \left\| \frac{x}{\|x\|_\alpha} \right\|_\beta =  \frac{\|x\|_{\beta}}{\|x\|_\alpha} \leq C_2 < \infty
\]
于是 $\| \cdot \|_\alpha$ 与 $\| \cdot \|_\beta$ 等价. 

又由于范数是任意的, 故有限维空间上任意两范数都等价, 命题得证.
\end{solution}

% =============== 题17 ===============
\question{}
证明: 在 \( \R^n \) 上, 当且仅当 \( A \) 是正定阵时函数 \( f(x) = (x^{\T}Ax)^{\frac{1}{2}} \) 是一个向量范数. 

\begin{solution}
从充分性和必要性两方面来证明.
\begin{enumerate}
    \item \textbf{必要性}: 若 \( f(x) \) 是向量范数, 则 \( A \) 正定. \\
    由向量范数的正定性知有 \( x^{\T}Ax \geq 0, \forall x \in \R^n \) 且 \( x^{\T}Ax = 0 \Leftrightarrow x = 0 \),
    这就是\( A \) 正定. 

    \item \textbf{充分性}: 若 \( A \) 正定, 则 \( f(x) \) 是向量范数.
    \begin{enumerate}
        \item \textbf{正定性}:
        由 \( A \) 正定, \( x^{\T}Ax > 0 \) 对所有 \( x \neq 0 \) 成立, 且 \( x^{\T}Ax = 0 \) 当且仅当 \( x = 0 \). 故
        \[ f(x) = (x^{\T}Ax)^{\frac{1}{2}} \geq 0 \quad \text{且} \quad f(x) = 0 \Leftrightarrow x = 0 \]

        \item \textbf{齐次性}:
        \( \forall \alpha \in \R \),
        \[ f(\alpha x) = ((\alpha x)^{\T}A(\alpha x))^{\frac{1}{2}} = (|\alpha|^2 x^{\T}Ax)^{\frac{1}{2}} = |\alpha| (x^{\T}Ax)^{\frac{1}{2}} = |\alpha| f(x) \]

        \item \textbf{三角不等式}:
        由Cauchy-Schwarz不等式, 结合$A$的对称正定性知
        \[
        \left(x^{\T}Ay \right)^2 = \langle A^{\frac{1}{2}}x, A^{\frac{1}{2}}y \rangle^2 \leq \|A^{\frac{1}{2}}x\|^2_2\|A^{\frac{1}{2}}y\|^2_2 = \left(x^{\T}Ax\right) \left(y^{\T}Ay\right)  
        \]
        同时有\begin{align*}
        &\quad \left[f(x) + f(y)\right]^2 - \left[f(x + y)\right]^2 \\
        &= \left[\left(x^{\T}Ax \right)^{\frac{1}{2}} + \left(y^{\T}Ay \right)^{\frac{1}{2}}\right]^2 - \left[\left((x + y)^{\T}A(x + y)\right)^{\frac{1}{2}}\right]^2  \\
        &= 2\sqrt{\left(x^{\T}Ax\right)\left(y^{\T}Ay\right)} - 2\left(x^{\T}Ay\right)
        \end{align*}
        又\(f\)非负且$A$正定, 则结合上述二式可知: \[ f(x) + f(y) \geq f(x + y) \]
    \end{enumerate}
    综上, $f(x)$确是向量范数.
\end{enumerate}
证毕.
\anothersolution{}
仅讨论充分性: 可以从对称性、双线性性、正定性三个方面证明$\phi(x, y) \triangleq x^{\T}Ay$是$\R^n$上的一个内积, 则$f(x) = \sqrt{\phi(x, x)}$自然就是$\R^n$上的一个向量范数.
\end{solution}

% =============== 题18 ===============
\question{}
若 $\|\cdot\|$ 是 $\R^m$ 上的一个向量范数. 证明: 若对于 $A \in \R^{m \times n}$, $\operatorname{rank}(A) = n$, 则
\(
\|x\|_A \triangleq \|A x\|
\)
是 $\R^n$ 上的一个向量范数. 

\begin{solution}
证明如下:
\begin{enumerate}
    \item \textbf{正定性}: 因为 $ \|\cdot\|$ 有正定性, 所以 $\|A x\| \geq 0$ 且 $\|A x\| = 0 \Leftrightarrow A x = 0$.

    又 $\operatorname{rank}(A) = n \text{且} A \in \R^{m \times n}$, 所以$A$列满秩, 即列向量间线性无关, 
    故 $A x = 0 \Longleftrightarrow x = 0$, 因此 $\|x\|_A = \|A x\| \geq 0$ 且 $\|x\|_A = 0 \Longleftrightarrow x = 0$. 

    \item \textbf{齐次性}: $\|\cdot\|$ 满足齐次性, $\forall \alpha \in \R$ 有
    \[
    \|\alpha x\|_A = \|A (\alpha x)\| = \|\alpha A x\| = |\alpha| \|A x\| = |\alpha| \|x\|_A
    \]

    \item \textbf{三角不等式}: $\forall x, y \in \R^n$, 有
    \[
    \|x + y\|_A = \|A(x + y)\| = \|A x + A y\| \leq \|A x\| + \|A y\| = \|x\|_A + \|y\|_A
    \]
\end{enumerate}
综上, $\|x\|_A$ 是 $\R^n$ 上的一个向量范数. 
\end{solution}

% =============== 题19 ===============
\question{}
设 $\|\cdot\|$ 是由向量范数 $\|\cdot\|$ 诱导出的矩阵范数. 证明: 若 $A \in \R^{n \times n}$ 非奇异, 则
\[
\|A^{-1}\|^{-1} = \min_{\|x\|=1} \|Ax\|
\]

\begin{solution}
证明如下:
\begin{enumerate}
    \item \textbf{是下界}: 首先, $\forall x \in \R^n \ \text{且}\ \|x\| = 1$有 \[ 1 = \|x\| = \|A^{-1} A x\| \leq \|A^{-1} \| \cdot \|Ax\| \Rightarrow \|A^{-1}\|^{-1} \leq \|Ax\| \]
    \item \textbf{可以取到}: 然后, 由 \[ \|A^{-1}\| = \max_{\|x\|=1} \|A^{-1} x\| \] 可知存在 $ x_1 \in \R^n \text{且} \|x_1\| = 1 $ 使得 \[ \|A^{-1}\| = \|A^{-1} x_1\| \]
    则现在只需要找出一个 $x_2 \in \R^n \text{且} \|x_2\| = 1$ 使得 \[ \|Ax_2\| = \|A^{-1}\|^{-1} = \frac{1}{\|A^{-1} x_1\|} \] 即可.
    而 \[ x_2 = \frac{A^{-1} x_1}{\|A^{-1} x_1\|} \] 符合题意, 故原命题得证.
    \end{enumerate}
\anothersolution{}
由诱导矩阵范数的定义, 有 
\[
\| A^{-1} \| = \max_{\| y \| = 1} \| A^{-1} y \|
\]
由于 \( A \) 非奇异, 映射 \( x \mapsto \frac{A x}{\| A x \|} \) 是从 \(\{x: \|x\|=1\}\) 到 \(\{y: \|y\|=1\}\) 的满射(显然此过程中有$x \neq 0, y \neq 0$). 因此
\[
\| A^{-1} \| = \max_{\| x \| = 1} \left\| A^{-1} \left( \frac{A x}{\| A x \|} \right) \right\| = \max_{\| x \| = 1} \frac{\| x \|}{\| A x \|} = \max_{\| x \| = 1} \frac{1}{\| A x \|}
\]
于是
\[
\| A^{-1} \| = \frac{1}{\min_{\| x \| = 1} \| A x \|}
\quad \Rightarrow \quad
\| A^{-1} \|^{-1} = \min_{\| x \| = 1} \| A x \|
\]
证毕. 
\end{solution}

% =============== 题20 ===============
\question{}
设 \( A = LU \) 是 \( A \in \R^{n \times n} \) 的 LU 分解. 其中 \( |l_{ij}| \leq 1 \). 又 \( a_i^{\T} \) 和 \( u_i^{\T} \) 分别表示 \( A \) 和 \( U \) 的第 \( i \) 行. 证明
\[
u_i^{\T} = a_i^{\T} - \sum_{j=1}^{i-1} l_{ij} u_j^{\T}
\]
且 \( \|U\|_\infty \leq 2^{n-1} \|A\|_\infty \).

\begin{solution}
因为
\[
A = \begin{pmatrix}
a_1^{\T} \\
a_2^{\T} \\
\vdots \\
a_n^{\T}
\end{pmatrix},\quad
L = \begin{pmatrix}
1 & 0 & 0 & \cdots & 0 \\
l_{21} & 1 & 0 & \cdots & 0 \\
l_{31} & l_{32} & 1 & \cdots & 0 \\
\vdots & \vdots & \vdots & \ddots & 0 \\
l_{n1} & l_{n2} & l_{n3} & \cdots & 1 \\
\end{pmatrix},\quad
U = \begin{pmatrix}
u_1^{\T} \\
u_2^{\T} \\
\vdots \\
u_n^{\T}
\end{pmatrix}
\]
结合 \( A = LU \) 于是有
\( a_i^{\T} = u_i^{\T} + \sum_{j=1}^{i-1} l_{ij} u_j^{\T} \)
, 即
\[
u_i^{\T} = a_i^{\T} - \sum_{j=1}^{i-1} l_{ij} u_j^{\T}
\]
下证明不等式部分(矩阵的无穷范数是其所有行向量的1-范数中最大者):
\[
i = 1, \|u_1^{\T}\|_1 = \|a_1^{\T}\|_1 \leq \|A\|_{\infty}
\]
\[
i = 2, \|u_2^{\T}\|_1 = \|a_2^{\T} - l_{21} u_1^{\T}\|_1 \leq (1+1) \|A\|_{\infty} = 2 \|A\|_{\infty}
\]
\[
i = 3, \|u_3^{\T}\|_1 = \|a_3^{\T} - l_{31} u_1^{\T} - l_{32} u_2^{\T}\|_1 \leq (1+1+2) \|A\|_{\infty} = 4 \|A\|_{\infty}
\]
\[
\vdots
\]
\[
i = n, \|u_n^{\T}\|_1 \leq (1+1+2+4+\cdots+2^{n-2}) \|A\|_{\infty} = 2^{n-1} \|A\|_{\infty}
\]
又
\[
\|U\|_{\infty} = \max_{1 \leq i \leq n}\|u_i^{\T}\|_1
\]
因此
\[
\|U\|_{\infty} \leq 2^{n-1} \|A\|_{\infty}
\]
证毕.
\end{solution}

% =============== 题21 ===============
\question{}
$A$ 和 \( A + E \) 都非奇异. 证明: 
\[
\|(A + E)^{-1} - A^{-1}\| \leq \|E\| \|A^{-1}\| \|(A + E)^{-1}\|
\]

\begin{solution}
注意到
\[ (A + E)^{-1} - A^{-1} = (A + E)^{-1} (E - (A + E) A^{-1}) = -(A + E)^{-1}EA^{-1} \]
直接取范数并由相容性即得
\[ \|(A + E)^{-1} - A^{-1}\| = \| (A + E)^{-1}EA^{-1} \| \leq \|E\| \|A^{-1}\| \|(A + E)^{-1}\| \]  
\end{solution}

% =============== 题22 ===============
\question{}
$\mathfrak{F} = \mathcal{F}(\beta, t, L, U)$为一个浮点数集合. $m = \beta^{L - 1}, M = \beta^U(1 - \beta^{-t})$.
设 \( m \leq |x| \leq M \).证明:
\[
\mathrm{fl}(x) = \frac{x}{1 + \delta},\quad |\delta| \leq u
\]
其中
\[
u = 
\begin{cases} 
\dfrac{1}{2} \beta^{1 - t}, & \text{用舍入法} \\ 
\beta^{1 - t}, & \text{用截断法}
\end{cases}
\]
为机器精度.

\begin{solution}
事实上\[\mathrm{fl}(x) = \dfrac{x}{1 + \delta} \text{ 与 }\ \mathrm{fl}(x) = x(1 + \delta)\]表意相同.
现不妨假定 \( x > 0 \)(因若 \( x < 0 \), 证明完全类似). 设 \( \alpha \) 是满足
\[
\beta^{\alpha - 1} \leq x < \beta^{\alpha}
\]
的唯一整数. 在 \( [\beta^{\alpha - 1}, \beta^{\alpha}) \) 中浮点数的阶为 \( \alpha \), 所以在这个区间中所有 \( t \) 位的浮点数以间距 \( \beta^{\alpha - t} \) 分布. 对于舍入法, 根据上式, 有
\[
|\mathrm{fl}(x) - x| \leq \frac{1}{2} \beta^{\alpha - t} = \frac{1}{2} \beta^{\alpha - 1} \beta^{1 - t} \leq \frac{1}{2} x \beta^{1 - t}
\]
即 
\[
\frac{|\mathrm{fl}(x) - x|}{x} \leq \frac{1}{2}\beta^{1-t}
\]
对于截断法, 有
\[
|\mathrm{fl}(x) - x| \leq \beta^{\alpha - t} = \beta^{\alpha - 1}\beta^{1 - t} \leq x\beta^{1 - t}
\]
即
\[
\frac{|\mathrm{fl}(x) - x|}{x} \leq \beta^{1 - t}
\]
\end{solution}

% =============== 题23 ===============
\question{}
$A \in \R^{n \times n}$存在 Jordan 分解
\[
X^{-1} A X = 
\begin{pmatrix}
\lambda_1 & \delta_1 & & & \\
& \lambda_2 & \delta_2 & & \\
& & \ddots & \ddots & \\
& & & \lambda_{n-1} & \delta_{n-1} \\
& & & & \lambda_n
\end{pmatrix}
\]
其中$X \in \C^{n \times n}$非奇异, \(\delta_i = 1\) 或 \(0\). \( \forall \varepsilon > 0 \), 令
\[
D_\varepsilon = \operatorname{diag}(1, \varepsilon, \varepsilon^2, \cdots, \varepsilon^{n-1})
\]
证明:
$ \left( 1 \right) $ $\|x\|_{XD_\varepsilon} \overset{\triangle}{=} \|(XD_{\varepsilon})^{-1}x\|_{\infty}$, $x \in \C^n$ 为一个向量范数.

$ \left( 2 \right) $ $\forall G \in \C^{n \times n}, \|G\|_{\varepsilon} \overset{\triangle}{=} \|D^{-1}_{\varepsilon}X^{-1}GXD_{\varepsilon}\|_{\infty}$ 为$\| \cdot \|_{XD_\varepsilon}$诱导出的算子范数.

\begin{solution}
$ \left( 1 \right) $ 记 $ B = XD_{\varepsilon} $, 证明如下:
\begin{enumerate}
    \item \textbf{正定性}: $\forall x \in \C^{n} $,有 \[ \|x\|_{XD_{\varepsilon}} = \|B^{-1}x\|_{\infty} \geq 0 \ \text{且} \ \|x\|_{XD_{\varepsilon}} = 0 \Leftrightarrow B^{-1}x = 0 \Leftrightarrow x = 0 \]
    \item \textbf{齐次性}: $\forall x \in \C^{n}, \forall \alpha \in \C$,有 \[ \|\alpha x\|_{XD_{\varepsilon}} = \|\alpha B^{-1}x\|_{\infty} = |\alpha| \|B^{-1}x\|_{\infty} = |\alpha| \|x\|_{XD_{\varepsilon}}\]
    \item \textbf{三角不等式}: $\forall x,y \in \C^{n} $,有 \[ \|x + y\|_{XD_{\varepsilon}} = \|B^{-1}x + B^{-1}y\|_{\infty} \leq \|B^{-1}x\|_{\infty} + \|B^{-1}y\|_{\infty} = \|x\|_{XD_{\varepsilon}} + \|y\|_{XD_{\varepsilon}} \]
\end{enumerate}
则$ \|x\|_{XD_\varepsilon} \overset{\triangle}{=} \|(XD_{\varepsilon})^{-1}x\|_{\infty} $确为向量范数. \\

$ \left( 2 \right) $ 同样记 $ B = XD_{\varepsilon} $, 
证明原命题等价于证明
\[
\|G\|_{\varepsilon} = \max_{\|x\|_{XD_\varepsilon} = 1} \|Gx\|_{XD_\varepsilon}
\]
即证
\[
\|B^{-1}GB\|_{\infty} = \max_{\|B^{-1}x\|_{\infty}= 1} \|B^{-1}Gx\|_{\infty}
\]
再记 $B^{-1}GB = M \in \mathbb{C}^{n \times n},\ B^{-1}x = y \in \C^n$, 则原命题可化为
\[
\|M\|_{\infty} = \max_{\|y\|_{\infty} = 1} \|My\|_{\infty}
\]
由于矩阵的$\infty$范数本身就是由向量的$\infty$范数如此诱导而来(同时结合$B^{-1}$是$\C^n \rightarrow \C^n$上的一个双射), 此式成立, 证毕.
\end{solution}

% =============== 题24 ===============
\question{}
若 $\|A\|<1$, 且 $\|I\|=1$. 证明: \[ \|(I-A)^{-1}\|\leq\frac{1}{1-\|A\|} \]

\begin{solution}
证明如下:
\begin{enumerate}
\item \textbf{先证$I-A$可逆}: 假设$I-A$是不可逆的, 则$\exists x \in \R^n$且$x \neq 0$, 
使得\[ (I-A)x=0 \Rightarrow x=Ax \] 即有\[0<\|x\|=\|Ax\|\leq\|A\|\cdot\|x\| \Rightarrow \|A\|\geq 1\]
矛盾, 则$I-A$可逆.

\item \textbf{再证不等式}: \[ I=(I-A)(I-A)^{-1}=(I-A)^{-1}-A(I-A)^{-1} \]
两边取范数, 则 \[ \|I\|=\|(I-A)^{-1}-A(I-A)^{-1}\| \]
由三角不等式和相容性有 \[ 1 \geq \|(I-A)^{-1}\|-\|A\|\cdot\|(I-A)^{-1}\| \overset{\|A\| < 1}{\Longrightarrow} \|(I-A)^{-1}\| \leq \frac{1}{1-\|A\|} \]
证毕.
\end{enumerate}
\anothersolution{} 仅证明$I-A$可逆: 
由 $\rho(A)\leq\|A\|<1$ 知 \(1\) 不是 \(A\) 的特征值, 即 \(I - A\) 可逆.
\end{solution}

% =============== 题25 ===============
\question{}
证明: 对任意的矩阵范数都有$\kappa(A) \geq 1$.

\begin{solution}
因为
\[
\|I\|=\|I \cdot I\|\le\|I\|\cdot\|I\|
\]
又
\[
\|I\| > 0 \Rightarrow \|I\|\ge1
\]
所以
\[
k(A)=\|A^{-1}\|\cdot\|A\|\geq\|A^{-1}A\|=\|I\|\geq1 \Rightarrow k(A)\geq1
\]
\end{solution}

% =============== 题26 ===============
\question{}
设 $A$ 为带状矩阵, 带宽为 $2m+1$, 其中 $m=3$. 若用列主元 Gauss 消去法计算得到的 $\widetilde{L}$ 和 $\widetilde{U}$ 满足
\[
\widetilde{L} \widetilde{U} = P(A + E)
\]
其中 $P$ 是排列方阵. 试估计 $\|E\|_{\infty}$ 的大小.

\begin{solution}
由课本中定理可知: 
\[
\|E\|_{\infty} \leq 4.09 \|A\|_{\infty} n^3 \rho u
\]
我们只需要估算增长因子 \(\rho\) 即可.

对于带宽为 \(2m+1\) 的带状矩阵(其中 \(m=3\)), 使用列主元 Gauss 消去法可知 \(\rho\) 满足: 
\[
\rho \leq 2^m = 2^3 = 8
\]
这一估计是基于消去过程中元素值的增长规律: 每一步消去可能使元素值最多翻倍, 并且由于每行最多涉及 \(m\) 次消去操作, 因此\(\rho\)以 \(2^m\) 为上界.

代入估计式得
\[
\|E\|_{\infty} \leq 4.09 \times 8 \|A\|_{\infty} n^3 u = 32.72 \|A\|_{\infty} n^3 u
\]
因此, \(\|E\|_{\infty}\) 的估计为 \(O(\|A\|_{\infty} n^3 u)\).
\end{solution}


\end{questions}

\end{document}