\documentclass[12pt, answers]{exam}     % 试卷
\usepackage{ctex}          % 中文支持
\usepackage{amsmath, amssymb, amsthm}       % 数学公式包
\usepackage{enumitem}
\usepackage{geometry}      % 设置页边距   
\usepackage{float}     
\usepackage[ruled, noline]{algorithm2e}  % 伪代码      
\geometry{a4paper, margin = 1.5cm}

% 定义环境
\SetKwProg{Func}{function:}{}{end}   % 伪代码样式
\newcommand{\anothersolution}{\par\noindent\textbf{另解:}}

% 自定义命令
\newcommand{\rank}{\operatorname{rank}}
\newcommand{\tr}{\operatorname{tr}}
\newcommand{\diag}{\operatorname{diag}}
\newcommand{\XLS}{\mathcal{X}_{\mathrm{LS}}}
\newcommand{\R}{\mathbb{R}}
\newcommand{\C}{\mathbb{C}}
\newcommand{\N}{\mathbb{N}}
\newcommand{\A}{\mathcal{A}}
\newcommand{\range}{\mathcal{R}}
\newcommand{\argmin}{\operatorname*{argmin}} 
\newcommand{\T}{\mathrm{T}}
\newcommand{\HH}{\mathrm{H}}

% 标题部分  
\title{数值线代の题库二}
\author{by 23大数据miracle}
\date{}

\begin{document}

\maketitle

\begin{questions}

% =============== 题1 ===============
\question{}
证明:\( x \in \XLS \) 当且仅当
\[
A^{\T} A x = A^{\T} b
\]

\begin{solution}
设 \( x \in \XLS \).  即
\[
x = \argmin_{x \in \R^n}  \left\{ \|b - Ax\|_2 \right\}
\]

令\( b = b_1 + b_2 \), 其中 \( b_1 \in \range(A) \) 且 \( b_2 \in \range(A)^{\perp} \).  则
有 \( Ax = b_1 \), 令
\[
r(x) = b - Ax = b - b_1 = b_2 \in \range(A)^{\perp}
\]
因此 \( A^{\T} r(x) = A^{\T} b_2 = 0 \).  将 \( r(x) = b - Ax \) 代入 \( A^{\T} r(x) = 0 \) 即得\( A^{\T} A x = A^{\T} b \). 

反之, 设 \( x \in \R^n \) 满足 \( A^{\T} A x = A^{\T} b \), 则对任意的 \( y \in \R^n \), 有
\begin{align*}
\|b - A(x + y)\|_2^2 &= \|b - Ax\|_2^2 - 2y^{\T} A^{\T} (b - Ax) + \|Ay\|_2^2 \\
&= \|b - Ax\|_2^2 + \|Ay\|_2^2 \\
&\geq \|b - Ax\|_2^2
\end{align*}
由此即得\( x \in \XLS \). 综上, 命题得证. 

\end{solution}


% =============== 题2 ===============
\question{}
证明 QR 分解定理, 即设 \( A \in \R^{m \times n} \) (\( m \geq n \)), 则 \( A \) 有 QR 分解:
\[
A = Q 
\begin{bmatrix}
R \\
0
\end{bmatrix}
\]
其中 \( Q \in \R^{m \times m} \) 是正交矩阵, \( R \in \R^{n \times n} \) 是具有非负对角元的上三角阵;而且当 \( m = n \) 且 \( A \) 非奇异时, 上述分解是唯一的. 

\begin{solution}
先证明 QR 分解的存在性. 对 \( n \) 用数学归纳法. 当 \( n = 1 \) 时, 此定理自然成立. 现假设已经证明定理对所有的 \( p \times (n-1) \) 矩阵成立, 这里假定 \( p \geq n-1 \). 
设 \( A \in \R^{m \times n} \) 的第一列为 \( a_1 \), 则由 Householder 变换相关知识知存在正交矩阵 \( Q_1 \in \R^{m \times m} \), 使得
\[
Q_1^{\T} a_1 = \|a_1\|_2 e_1 
\]
于是, 有
\[
Q_1^{\T} A = 
\begin{bmatrix}
\|a_1\|_2 & v^{\T} \\
0 & A_1
\end{bmatrix}
\]
对 \( (m-1) \times (n-1) \) 矩阵 \( A_1 \) 应用归纳法假设, 得
\[
A_1 = Q_2 
\begin{bmatrix}
R_2 \\
0
\end{bmatrix}
\]
其中 \( Q_2 \) 是 \( (m-1) \times (m-1) \) 正交矩阵, 而 \( R_2 \) 是具有非负对角元的 \( (n-1) \times (n-1) \) 上三角阵. 这样, 令
\[
Q = Q_1 
\begin{bmatrix}
1 & 0 \\
0 & Q_2
\end{bmatrix}, \quad
R = 
\begin{bmatrix}
\|a_1\|_2 & v^{\T} \\
0 & R_2 
\end{bmatrix}
\]
则有
\[
Q \begin{bmatrix} R \\ 0 \end{bmatrix} = 
Q_1 \begin{bmatrix} 1 & 0 \\ 0 & Q_2 \end{bmatrix} \begin{bmatrix} \|a_1\|_2 & v^{\T} \\ 0 & R_2 \\ 0 & 0 \end{bmatrix}
= Q_1 Q_1^{\T} A = A
\]

即 \( Q \) 与 \( R \) 满足定理的要求. 于是, 由归纳法原理知存在性得证. 

再证唯一性. 设 \( m = n \) 且 \( A \) 非奇异, 并假定 \( A = QR = \tilde{Q}\tilde{R} \), 其中 \( Q, \tilde{Q} \in \R^{m \times m} \) 是正交矩阵, 
\( R, \tilde{R} \in \R^{n \times n} \) 是具有非负对角元的上三角阵. \( A \) 非奇异蕴涵着 \( R \) 和 \( \hat{R} \) 的对角元均为正数. 因此, 我们有
\[
\tilde{Q}^{\T} Q = \tilde{R} R^{-1}
\]
此矩阵既是正交矩阵又是对角元均为正数的上三角阵, 故只能是单位矩阵, 从而必有 \( \tilde{Q} = Q \), \( \tilde{R} = R \), 即分解是唯一的. 
\end{solution}  

% =============== 题3 ===============
\question{}
假设 \( x \) 和 \( y \) 是 \( \R^n \) 中的两个单位向量. 给出一种使用 Givens 变换的算法, 计算一个正交矩阵 \( Q \), 使得 \( Qx = y \). 

\begin{solution}
核心思路: 使用 Givens 变换分别将 \( x \)和\( y \) 旋转到 \( e_1 \), 得到正交矩阵 \( Q_1 \)和\( Q_2 \) 满足 \( Q_1x = e_1 \)与\(Q_2y = e_1 \),
于是便有 \( Q = Q_2^{\T} Q_1 \) 满足 \( Qx = y \).

以下是后面各函数的简要说明: \\
1. \texttt{givens}($a, b$):
   计算单次 Givens 变换的旋转参数 \( c, s \). 

2. \texttt{rotVec}($v, i, j, c, s$):
   对向量 \( v \) 的第 \( i, j \) 分量应用 Givens 变换(参数为 \( c, s \)). 

3. \texttt{rotMat}($Q, i, j, c, s$):
   对矩阵 \( Q \) 的第 \( i, j \) 行应用 Givens 变换(参数为 \( c, s \)), 来记录总变换. 

4. \texttt{rotFstAx}($v$):
   用 Givens 旋转将向量 \( v \) 逐步对齐到 \( e_1 \)(主坐标轴), 返回累积正交矩阵 \( U \). 

5. \texttt{main}($x, y$):
   主函数, 即计算最后的正交矩阵 \( Q \). 
\\
\begin{algorithm}[H]
\SetKwFunction{FGivens}{[c, s]=givens}
\Func{\FGivens{$a, b$}}{
    \eIf{$|b| < \epsilon$}{
        $c = \operatorname{sign}(a)$; $s = 0$ 
    }{
        \eIf{$|a| < \epsilon$}{
            $c = 0$; $s = \operatorname{sign}(b)$ 
        }{
            $r = \sqrt{a^2 + b^2}$;
            $c = a/r$; $s = b/r$ 
        }
    }
}
--------------------------------------------------------------------------------------------------------------\\
\SetKwFunction{FRotVec}{[v]=rotVec}
\Func{\FRotVec{$v, i, j, c, s$}}{
    $\sigma = c \cdot v_i + s \cdot v_j$\\
    $v_j = -s \cdot v_i + c \cdot v_j$\\
    $v_i = \sigma$\\
}
--------------------------------------------------------------------------------------------------------------\\
\SetKwFunction{FRotMat}{[Q]=rotMat}
\Func{\FRotMat{$Q, i, j, c, s$}}{
    \For{$k = 1 : n$}{
        $\tau = c \cdot Q_{i, k} + s \cdot Q_{j, k}$\\
        $Q_{j, k} = -s \cdot Q_{i, k} + c \cdot Q_{j, k}$\\
        $Q_{i, k} = \tau$\\
    }
}
--------------------------------------------------------------------------------------------------------------\\
\SetKwFunction{FRotFstAx}{[U]=rotFstAx}
\Func{\FRotFstAx{$v$}}{
    $U = I_n$  \\
    \For{$k = n : 2$}{
        $i = k-1$; $j = k$\\
        $(c, s) = \texttt{givens}(v_i, v_j)$\\
        $v = \texttt{rotVec}(v, i, j, c, s)$\\
        $U = \texttt{rotMat}(U, i, j, c, s)$\\    }
}
--------------------------------------------------------------------------------------------------------------\\
\SetKwFunction{FMain}{[Q]=main}
\Func{\FMain{$x, y$}}{
    $Q_1 = \texttt{rotFstAx}(x)$; $Q_2 = \texttt{rotFstAx}(y)$
    
    $Q = Q_2^{\T} Q_1$
}
\caption{用 Givens 算法计算正交阵$Q$, 满足$Qx = y$, 且$\| x \|_2 = \| y \|_2 = 1$}
\end{algorithm}

其中关于该算法正确性的具体证明:
\begin{align*}
Qx &= (Q_2^{\T} Q_1) x \\
&= Q_2^{\T} (Q_1 x) \\
&= Q_2^{\T} e_1 \\
&= Q_2^{\T} (Q_2 y) \\  
&= y
\end{align*}
证毕. 
\end{solution}


% =============== 题4 =============== 
\question{}设 \( x \) 和 \( y \) 是 \( \R^n \) 中的两个非零向量. 给出一种算法来确定一个 Householder 变换 \( H \), 使得 \( Hx = \alpha y \), 其中 \( \alpha \in \R \). 

\begin{solution}
当 \( x \) 和 \( y \)线性相关时, 直接取\( H = I\). 否则按照以下思路计算出$H$即可:
\[
H = I - 2w w^{\T} \Rightarrow Hx = x - 2 w (w^{\T}x) = \alpha y
\]
则有$w \ \text{平行于} \ (x - \alpha y)$, 又$\|w\|_2 = 1$, 故取
\[
w = \frac{x - \alpha y}{\|x - \alpha y\|_2}
\]
具体算法如下:\\
\begin{algorithm}[H]
\SetKwFunction{FHouse}{[H]=house}
\Func{\FHouse{$x, y$}}{
    $\alpha = \|x\|_2 / \|y\|_2$ \\
    $z = x - \alpha y$ \\
    $w = z / \| z \|_2$ \\
    $H = I - 2ww^{\T}$
}
\caption{用 Householder 变换求得对应矩阵$H$, 满足$Hx = \alpha y$, $\alpha \in \R$}
\end{algorithm}

其中关于该算法正确性的具体证明(此时\(z \ne 0\)): \\
展开有\begin{align*}
\|z\|_2^2 &= (x^{\T} - \alpha y^{\T})(x - \alpha y) \\
&= (x^{\T} - \alpha y^{\T})x + \alpha (\alpha y^{\T} - x^{\T})y  \\
&= z^{\T}x + x^{\T}x - \alpha x^{\T}y  \\
&= 2(z^{\T}x)
\end{align*}
于是
\begin{align*}
Hx &= x - 2ww^{\T}x \\
&= x - 2\left(\frac{z}{\|z\|_2}\right)\left(\frac{z^{\T}x}{\|z\|_2}\right) \\
&= x - \frac{2(z^{\T}x)}{\|z\|_2^2}z \\
&= x - z \\
&= x - (x - \alpha y)\\
&= \alpha y
\end{align*}
证毕. 
\end{solution}

% =============== 题5 ===============
\question{}设 \( A \in \R^{m \times n} \), 且存在 \( X \in \R^{n \times m} \) 使得对于每一个 \( b \in \R^m \), \( x = Xb \) 均极小化 \( \|Ax - b\|_2 \). 
证明:\( AXA = A \) 和 \( (AX)^{\T} = AX \). 

\begin{solution}
设 \( R(A) \) 为线性映射 \( f(x) = Ax \) 的值域, 则 \( R(A) \) 为 \( \R^m \) 的线性子空间. 
\begin{enumerate}
    \item
    考虑 \( b \in R(A) \), 这时存在 \( y \in \R^n \) 使得 \( Ay = b \). 因此 \( \|Ax - b\|_2 \) 最小值为 0, \( Ax - b = (AX - E)b = 0 \). 依次取 \( b_1, b_2, \cdots, b_n \) 为 \( A \) 的列向量, 则有 \( (AX - E)b_i = 0 \), \( i = 1, 2, \cdots, n \). 即
    \[
    (AX - E)A = 0
    \]
    由此得
    \[
    AXA = A
    \]
    \item
    对于所有 \( b \in \R^m \), 由于 \( \|Ax - b\|_2 \) 取到极小值, 由课本定理可知
    这时 \( b - Ax \) 与 \( Ax \) 互相垂直, 即
    \[
    (Ax)^{\T}(Ax - b) = 0
    \]
    再将 \( x = Xb \) 代入, 有
    \[
    (AXb)^{\T}(AXb - b) = b^{\T} (AX)^{\T} (AX - E) b = 0
    \]
    由于 \( b \) 的任意性, 可知
    \[
    (AX)^{\T} (AX - E) = 0
    \]
    因此, 
    \[
    (AX)^{\T} AX = (AX)^{\T}
    \]
    两边取转置得:
    \[
    (AX)^{\T} AX = AX
    \]
    比较上两式, 即有
    \[
    (AX)^{\T} = AX
    \]
\end{enumerate}

\anothersolution{}因为 $ x $ 极小化 \( \|Ax - b\|_2 \), 所以 $ \forall b \in \R^m $, 
有 $ A^{\T}b = A^{\T}Ax = A^{\T}AXb $, 于是
\[
(A^{\T}AX - A^{\T})b = 0
\]
依次取 $ b = e_1 , e_2, \dots, e_n $, 有
\[
(A^{\T}AX - A^{\T})(e_1, e_2, \dots, e_n) = (A^{\T}AX - A^{\T})I = 0 \implies A^{\T}AX = A^{\T}  
\]
于是
\[
A = X^{\T}A^{\T}A \implies AX = X^{\T}A^{\T}AX = (AX)^{\T}AX \text{ 对称}
\]  
则有 $ AX $ 对称, 即
\[
(AX)^{\T} = AX
\]
进而
\[
AXA = (AX)^{\T}A = X^{\T}A^{\T}A = A
\]
\end{solution}


% =============== 题6 ===============   
\question{}利用等式
\[
\|A(x + \alpha w) - b\|_2^2 = \|Ax - b\|_2^2 + 2\alpha w^{\T} A^{\T} (Ax - b) + \alpha^2 \|Aw\|_2^2
\]
证明:如果 \( x \in \XLS \), 那么 \( A^{\T} A x = A^{\T} b \). 

\begin{solution}
定义泛函:
\[
f(x) = \|Ax - b\|_2^2
\]
对于任意 \(x \in \mathcal{X}_{LS}\), \(x\) 是 \(f(x)\) 的一个极小值点. 由于 \(f\) 是二次函数, 其在 \(\R^n\) 上处处可微(特别地, 任意方向导数均存在), 且定义域为开集, 故 \(x\) 作为内点极小值点, 其方向导数在任意方向为零. 

考虑任意方向向量 \(w \in \R^n\). 方向导数在方向 \(w\) 上定义为:
\[
\lim_{\alpha \to 0} \frac{f(x + \alpha w) - f(x)}\alpha
\]
利用题中等式有:
\[
\frac{f(x + \alpha w) - f(x)}\alpha = 2 w^{\T} A^{\T} (Ax - b) + \alpha \|Aw\|_2^2
\]
取极限 \(\alpha \to 0\):
\[
\lim_{\alpha \to 0} \frac{f(x + \alpha w) - f(x)}\alpha = 2 w^{\T} A^{\T} (Ax - b)
\]
由方向导数为零:
\[
w^{\T} A^{\T} (Ax - b) = 0
\]
此式对所有 \(w \in \R^n\) 成立. 若固定 \(v = A^{\T} (Ax - b)\), 则 \(w^{\T} v = 0\) 对所有 \(w\) 成立. 取 \(w = v\), 得:
\[
v^{\T} v = \|v\|_2^2 = 0
\]
故 \(v = 0\), 即:
\[
A^{\T} A x = A^{\T} b
\]
\anothersolution{}注意到, $ \forall \alpha \in \R $有
\begin{align*}
0 &\le \|A(x + \alpha w) - b\|_2^2 - \|Ax - b\|_2^2  \\ 
&= 2\alpha w^{\T} A^{\T} (Ax - b) + \alpha^2 \|Aw\|_2^2  \\
&= \|Aw\|_2^2 \alpha^2 + 2w^{\T} A^{\T} (Ax - b) \alpha
\end{align*}
于是, 考虑其对应一元二次方程的判别式
\[
\Delta = 4[w^{\T} A^{\T} (Ax - b)]^2 - 4 \cdot \|Aw\|_2^2 \cdot 0 \le 0
\]
即
\[
w^{\T} A^{\T} (Ax - b) = 0 \overset{w \text{的任意性}}{\implies} A^{\T}Ax = A^{\T}b
\]
\end{solution}

% =============== 题7 ===============
\question{}若迭代矩阵 \( M \) 的范数 \(\|M\| = q < 1\), 则迭代法 $ x_k = Mx_{k-1} + g $ 所产生的近似解 \( x_k \) 与准确解 \( x_* \) 的误差有如下估计式:
\[
\|x_k - x_*\| \leq \frac{q^k}{1-q} \|x_1 - x_0\|, \quad \|x_k - x_*\| \leq \frac{q}{1-q} \|x_{k-1} - x_k\|
\]

\begin{solution}以下分别对两个估计式进行证明:
\begin{enumerate}
    \item
    令 $ y_k = x_k - x_* $ 并结合 $ x_* = Mx_* + g $, 有
    \[
    y_k = My_{k-1} \overset{\text{迭代}}{\implies}  y_k = M^k y_0
    \]
    两边取范数, 得
    \[
    \|y_k\| = \|M^k y_0\| \leq \|M\|^k \|y_0\| = q^k \|y_0\|
    \]
    现在估计 \( y_0 \). 根据定义, 我们有
    \begin{align*}
    \|y_0\| &= \|x_0 - x_*\| \\
    &\leq \|x_0 - x_1\| + \|x_1 - x_*\| \\
    &= \|x_0 - x_1\| + \|My_0\| \\
    &\leq \|x_0 - x_1\| + q \|y_0\|
    \end{align*}
    从而有
    \[
    \|y_0\| \leq \frac{1}{1-q} \|x_0 - x_1\|
    \]
    将此不等式代入 $ \|y_k\| \leq q^k \|y_0\| $, 即有:
    \[
    \|x_k - x_*\| \leq \frac{q^k}{1-q} \|x_1 - x_0\|
    \]
    \item
    因为
    \begin{align*}
    \|x_k - x_*\| &= \| M(x_{k-1} - x_*) \| \leq q \| x_{k-1} - x_* \|  \\
    &\leq q \| x_{k-1} - x_k \| + q \| x_k - x_* \|
    \end{align*}
    所以有
    \[
    \|x_k - x_*\| \leq \dfrac{q}{1-q} \|x_{k-1} - x_k\|
    \]
\end{enumerate}
\anothersolution{}
第二个估计式同上, 下证明第一个估计式:

使用数学归纳法, 首先当 $k=1$ 时, 需证
\[
\|x_1 - x_*\| \leq \frac{q}{1-q}\|x_1 - x_0\|
\]
这就是第二个估计式在$k=1$时的情形. 

然后假设在 $k-1$ 下估计式成立, 即 
\[
\|x_{k-1} - x_*\| \leq \frac{q^{k-1}}{1 - q} \|x_1 - x_0\|
\]
于是
\[
\|x_k - x_*\| = \|M (x_{k-1} - x_*)\| \leq q \cdot \frac{q^{k-1}}{1 - q} \|x_1 - x_0\| = \frac{q^k}{1 - q} \|x_1 - x_0\|
\]
由归纳假设知第一个估计式得证, 证毕. 
\end{solution}


% =============== 题8 ===============
\question{}若线性方程组 \( Ax = b \) 的系数矩阵 \( A \) 对称, 而且其对角元 \( a_{ii} > 0 \ (i = 1, \cdots, n) \), 
则 Jacobi 迭代法收敛的充分必要条件是 \( A \) 和 \( 2D - A \) 都正定. 
\begin{solution}
记 \( B = D^{-1}(L + U) = D^{-1}(D - A) = I - D^{-1}A \), 而 \( D = \text{diag}(a_{ii}) \) 的对角元均大于零, 故
\[
B = I - D^{-1}A = D^{-\frac{1}{2}}(I - D^{-\frac{1}{2}}AD^{-\frac{1}{2}})D^{\frac{1}{2}}. 
\]

由 \( A \) 的对称性推出 \( I - D^{-\frac{1}{2}}AD^{-\frac{1}{2}} \) 也是实对称的, 特征值均为实数, 而且它与 \( B \) 相似, 有相同的特征值, 从而 \( B \) 的特征值均为实数. 此外, 由上式立即可得
\[
I - B = D^{-\frac{1}{2}}(D^{-\frac{1}{2}}AD^{-\frac{1}{2}})D^{\frac{1}{2}}, 
\]
\[
I + B = D^{-\frac{1}{2}}(2I - D^{-\frac{1}{2}}AD^{-\frac{1}{2}})D^{\frac{1}{2}}. 
\]
又
\begin{align*}
\text{Jacobi 迭代法收敛} &\Leftrightarrow \rho(B) < 1 \text{, 即矩阵} B \text{的所有特征值} \lambda \text{满足} |\lambda| < 1 \\
&\Leftrightarrow I - B \text{和} I + B \text{的特征值均为正实数} \left( B\text{的特征值均为实数} \right) \\
&\Leftrightarrow D^{-\frac{1}{2}}AD^{-\frac{1}{2}} \text{和} 2I - D^{-\frac{1}{2}}AD^{-\frac{1}{2}} \text{的特征值均为正实数} \\
&\Leftrightarrow A \text{和} 2D - A \text{均正定}
\end{align*}
从而定理得证. 
\end{solution}


% =============== 题9 ===============
\question{}若矩阵 \( A \) 是严格对角占优的或不可约对角占优的, 则 \( A \) 非奇异且则 Jacobi 迭代法和 G-S 迭代法都收敛. 
\begin{solution}以下分两部分进行证明:
\begin{enumerate}
    \item
    对两种矩阵非奇异性的证明:   \\
    \noindent\textbf{当 $ A $ 为严格对角占优时:}用反证法. 假设 \( A \) 奇异, 则齐次方程组 \( Ax = 0 \) 有非零解 \( x \). 不妨取 \( |x_i| = \|x\|_{\infty} = 1 \), 则有
    \[
    |a_{ii}| = |a_{ii}x_i| = \left| \sum_{j=1}^{n} a_{ij}x_j \right| \leq \sum_{\substack{j=1 \\ j \neq i}}^{n} |a_{ij}|. 
    \]
    这与 \( A \) 严格对角占优矛盾, 因此\( A \) 非奇异. 

    \noindent\textbf{当 $ A $ 为不可约对角占优时:}仍用反证法. 设 \( x \) 满足 \(\|x\|_{\infty} = 1\), 使得 \( Ax = 0 \), 并定义
    \[
    \mathcal{S} = \{i: |x_i| = 1\}, 
    \]
    \[
    \mathcal{T} = \{k: |x_k| < 1\}. 
    \]
    显然 \( \mathcal{S} \cup \mathcal{T} = \mathcal{W} = \{1, 2, \cdots, n\} \), \( \mathcal{S} \cap \mathcal{T} = \emptyset \), 而且 \( \mathcal{T} \) 非空. 这是因为:假设 \( \mathcal{T} \) 为空集, 则 \( x \) 的各个分量的绝对值均为 1, 那么不论 \( i \) 为 \( \mathcal{S} \) 中的何值均有
    \[
    |a_{ii}| \leq \sum_{\substack{j=1 \\ j \neq i}}^{n} |a_{ij}|. 
    \]
    这与 \( A \) 弱严格对角占优矛盾. 因此, 在\(\mathcal{T}\)非空的情况下, 因为 \( A \) 不可约, 必定存在 \( i, k \), 使得
    \[
    a_{ik} \neq 0, \quad i \in \mathcal{S}, k \in \mathcal{T}. 
    \]
    于是, \( |a_{ik}x_k| < |a_{ik}| \), 并且
    \begin{align*}
    |a_{ii}| &\leq \sum_{j \in \mathcal{S}, j \neq i} |a_{ij}| |x_j| + \sum_{j \in \mathcal{T}} |a_{ij}| |x_j|  \\
    &< \sum_{j \in \mathcal{S}, j \neq i} |a_{ij}| + \sum_{j \in \mathcal{T}} |a_{ij}|  \\ 
    &= \sum_{j \neq i} |a_{ij}|. 
    \end{align*}
    这又与 \( A \) 弱严格对角占优矛盾, 则此时\( A \)同样非奇异. 
    \item
    对两种迭代法收敛性的证明:   \\
    若 \( A \) 是严格对角占优的或不可约对角占优的, 则对每个 \( i \), 必有 \( |a_{ii}| > 0 \). 因此 \( D \) 可逆. 

    \noindent\textbf{Jacobi 迭代法收敛性证明:}
    假设 Jacobi 迭代矩阵 \( B_1 = D^{-1}(L+U) \) 的某个特征值 \( |\lambda| \geq 1 \). 考察矩阵 \( \lambda D - L - U \), 它满足与原矩阵 \( A \) 相同的占优条件(严格对角占优或不可约对角占优), 故 \( \lambda D - L - U \) 非奇异. 由
    \[
    \lambda I - B_1 = \lambda I - D^{-1}(L + U) = D^{-1}(\lambda D - L - U)
    \]
    可得
    \[
    \det(\lambda I - B_1) = \det(D^{-1}) \det(\lambda D - L - U) \neq 0
    \]
    这与 \( \lambda \) 是 \( B_1 \) 的特征值矛盾. 因此 \( B_1 \) 的所有特征值模均小于 1, Jacobi 迭代法收敛. 

    \noindent\textbf{G-S 迭代法收敛性证明:}
    设 G-S 迭代矩阵为 \( B_2 = (D - L)^{-1}U \). 假设存在特征值 \( \lambda \) 满足 \( |\lambda| \geq 1 \). 考察矩阵:
    \[
    \lambda D - \lambda L - U = \lambda (D - L) + [A - (D - L)]
    \]
    同样的, 与矩阵 \( A \) 有相同的占优条件, 因此 \( \lambda D - \lambda L - U \) 非奇异, 
    于是, 
    \begin{align*}
    \det(\lambda I - B_2) &= \det\left( \lambda I - (D - L)^{-1}U \right)  \\
    &=\det ( (D - L)^{-1} ) \det ( \lambda D - \lambda L - U)  \\ &\ne 0
    \end{align*}

    与$\lambda$是$B_2$的特征值矛盾. 因此 \( B_2 \) 的所有特征值模均小于 1, G-S 迭代法收敛. 
\end{enumerate}
综上, 所有命题皆得证. 
\end{solution}

% =============== 题10 ===============
\question{}设 \( B \in \R^{n \times n} \) 满足 \(\rho(B) = 0\). 证明对于任意的 \( g, x_0 \in \R^n \), 
迭代格式 \( x^{(k+1)} = Bx^{(k)} + g \), \( k = 0, 1, 2, \ldots \) 最多迭代 \( n \) 次就可以得到方程组 \( x = Bx + g \) 的精确解. 

\begin{solution}
由于 \( B \) 谱半径为零, 故特征值都为零. 于是存在可逆矩阵 \( Q \) 和矩阵
\[
J =
\begin{bmatrix}
0 & \alpha_1 & * & \cdots & * \\
 & 0 & \alpha_2 & \ddots & \vdots \\
 & & \ddots & \ddots & * \\
 & & & 0 & \alpha_{n-1} \\
 & & & & 0
\end{bmatrix}
\]
使得 \( B = Q^{-1} JQ \). 注意到 \( J^n = 0 \), 因此 \( B^n = Q^{-1} J^nQ = 0 \). 这说明 \( y_n = B^n y_0 = 0 \). \\
即 \( x_n = x_* \) $ \Leftrightarrow$ 收敛到精确解.
\anothersolution{}
方程 $x = Bx + g$ 等价于 $(I - B)x = g$. 因 $B$ 是幂零矩阵, 则$I - B$ 可逆, 于是该方程的解存在唯一, 记为 $x_*$. \\
由迭代格式有
\[
y_{k+1} = B y_k \Rightarrow y_k = B^k y_0 
\]
考虑核空间升链, 于是
\[
\{0\} = \ker(B) \subseteq \ker(B^2) \subseteq \cdots \subseteq \ker(B^m) = \R^n
\]
其中 $m$ 为幂零指数(即$B^m=0$). 令 $d_k = \dim \ker(B^k)$, 则
\[
0 = d_0 \leq d_1 \leq \cdots \leq d_m = n
\]
事实上, 对于任意矩阵$B$, 若存在$p \in \N$使得$\ker(B^p) = \ker(B^{p+1})$,
则有 \[\ker(B^p) = \ker(B^{p+1}) = \cdots = \ker(B^s), \: \forall s \geq p \]
于是存在$0 < p \leq m$, 使得
\[
0 = d_0 < \cdots < d_p = \cdots = d_m = n
\]
而前面递增的部分每步维数至少增加 $1$, 故
\[
n = d_m \geq d_0 + p = p \Rightarrow n \geq p
\]
因而 $\ker(B^n) = \ker(B^p) = \ker(B^m) = \R^n$, 于是
\[
y_n = B^n y_0 = 0 \Rightarrow x_n = x_*
\]
补充对上述所用定理的证明:

已知存在 \( p \) 满足 \( \ker(B^p) = \ker(B^{p+1}) \), 证明:
\[ \ker(B^k) = \ker(B^p),\: \forall k > p \]

注意到事实上只需证明\( \ker(B^{p+1}) \supseteq \ker(B^{p+2}) \)即可:

考虑 \( \forall x \in \ker(B^{p+2}) \), 即 \( B^{p+2}x = 0 \), 则 \( B^{p+1}(Bx) = 0 \), 故 \( Bx \in \ker(B^{p+1}) = \ker(B^p) \). 于是 \( B^p(Bx) = B^{p+1}x = 0 \), 即 \( x \in \ker(B^{p+1}) \), 自然便有$ \ker(B^{p+1}) \supseteq \ker(B^{p+2}) $, 证毕.
\end{solution}


% =============== 题11 ===============
\question{}设 $A$ 是具有正对角元素的非奇异对称矩阵. 证明: 若求解方程组 $Ax = b$ 的 G-S 迭代法对任意初始近似皆收敛, 则 $A$ 必定是正定的. 

\begin{solution}
设 $A$ 为对称非奇异矩阵, 且对角元全为正. 令
\[
A = D - L - L^{\T}
\]
其中 $D$ 为全为正对角元的对角矩阵, $L$ 为严格下三角矩阵. 

G-S迭代格式为
\[
x_{k+1} = D^{-1}L x_{k+1} + D^{-1}L^{\T} x_k + D^{-1}b
\]
设精确解为 $x_*$, 于是有
\[
x_* = D^{-1}L x_* + D^{-1}L^{\T} x_* + D^{-1}b
\]
定义误差向量
\[
y_k = x_k - x_*
\]
则误差递推关系为
\[
y_{k+1} = D^{-1}L y_{k+1} + D^{-1}L^{\T} y_k
\]
即
\[
(D - L)y_{k+1} = L^{\T} y_k
\]
具体证明:
\begin{enumerate}
\item \textbf{引理推导}:
定义
\[
\epsilon_k = y_k - y_{k+1}
\]
可推得
\[
A y_{k+1} = D y_{k+1} - L y_{k+1} - L^{\T} y_{k+1}= L^{\T} \epsilon_k
\]
进一步有(优先消去$y_k$, 则比较容易推出)
\begin{align*}
y_k^{\T} A y_k - y_{k+1}^{\T} A y_{k+1} &= (y_{k+1} + \epsilon_k)^{\T} A (y_{k+1} + \epsilon_k) - y_{k+1}^{\T} A y_{k+1}  \\
&= \epsilon_k^{\T}A y_{k+1} + y_{k+1}^{\T}A \epsilon_k + \epsilon_k^{\T} A \epsilon_k \\
&= \epsilon_k^{\T} L^{\T} \epsilon_k + \epsilon_k^{\T} L \epsilon_k + \epsilon_k^{\T} A \epsilon_k \\
&= \epsilon_k^{\T} D \epsilon_k
\end{align*}
由于 $D$ 正定, 故当 $\epsilon_k \ne 0$ 时, 
\[
\epsilon_k^{\T} D \epsilon_k > 0
\]
即
\[
y_k^{\T} A y_k > y_{k+1}^{\T} A y_{k+1}
\]
说明未收敛时 $y_k^{\T} A y_k$ 严格递减. 
\item \textbf{反证得结论}: 假设 $A$ 不是正定的, 则存在非零向量 $x_0$ 使得
\[
x_0^{\T} A x_0 \leq 0
\]
令 $b = 0$(当 $b \neq 0$ 时, 由于 $A$ 非奇异, 线性方程组 $Ax = b$ 也等价于 $A(x - A^{-1}b) = 0$, 则作变量代换 $z = x - A^{-1}b$ 得 $Az = 0$), 
则精确解 $x_* = 0 \ \text{且}\ y_k = x_k$. 

但由前面的引理, 同时结合$y_0 = x_0 \neq 0$(即初始向量非最终精确解), 有
\[
y_k^{\T} A y_k \leq y_1^{\T} A y_1 < y_0^{\T} A y_0 \leq 0,\ \forall k \in \N
\]
故存在 $\epsilon' = y_1^{\T} A y_1 $ 使得
\[
y_k^{\T} A y_k \leq \epsilon' < 0,\ \forall k \in \N
\]
另一方面, 由G-S迭代收敛性假设应有($\lambda$指矩阵$A$模长最大的特征值):
\[
\| y_k^{\T} A y_k \|_2 \leq |\lambda| \| y_k \|_2^2 = |\lambda| \| x_k \|_2^2 \rightarrow 0,\ k \rightarrow \infty
\]
即
\[
\lim_{k \to \infty} y_k^{\T} A y_k = 0
\]
矛盾, 故假设不成立, $A$ 必为正定矩阵. 
\end{enumerate}
\end{solution}

% =============== 题12 ===============
\question{}若存在对称正定矩阵 \( P \), 使得
\[
B = P - H^{\T} P H
\]
为对称正定矩阵, 求证:迭代法
\[
x_{k+1} = H x_k + b, \quad k = 0, 1, 2, \cdots
\]
收敛. 

\begin{solution}
设 \(\lambda\) 为 \(H\) 的特征值, \(x\) 为 \(H\) 属于 \(\lambda\) 的一个特征向量. 则(结合$H$一定是实的)
\[
x^{\HH} B x = x^{\HH} P x - (H x)^{\HH} P (H x) = (1 - |\lambda|^{2}) x^{\HH} P x. 
\]
不妨设$x = u + iv$, 则一定有$u \neq 0$或者$v \neq 0$, 于是(结合$B$的对称正定性)
\[
x^{\HH} B x = (u^{\T} - iv^{\T})B(u + iv) = u^{\T} B u + v^{\T} B v > 0
\]
同理 \(x^{\HH} P x > 0\), 从而 \(1 - |\lambda|^{2} > 0\), 
即 \( |\lambda| < 1 \Leftrightarrow \rho(H) < 1\)($\lambda$是任意的). 
因此迭代法 \(x_{k+1} = H x_k + b\) 收敛. 
\end{solution}


% =============== 题13 ===============
\question{}证明: 若系数矩阵 \( A \) 是严格对角占优的或不可约对角占优的, 
且松驰因子 \( \omega \in (0, 1) \), 则 SOR 迭代法收敛. 

\begin{solution}
用反证法, 现假定某个复数 \( |\lambda| \geq 1 \)是系数矩阵\( A \)的SOR迭代矩阵\( L_{\omega} = (D - \omega L)^{-1} [(1 - \omega)D + \omega U] \)的特征值(由\( A \)的对角占优性可知\(L_{\omega}\)一定存在), 
不妨令 \( \lambda = \alpha + \beta i \), 则有 \( \alpha^2 + \beta^2 \geq 1 \), 
考虑矩阵\( (\lambda + \omega - 1)D - \lambda \omega L - \omega U \), 于是有:
\begin{align*}
|\lambda + \omega - 1|^2 - |\lambda|^2 \omega^2 &= (\alpha + \omega - 1)^2 + \beta^2 - \omega^2 (\alpha^2 + \beta^2) \\
&= \alpha^2 - 2\alpha (1 - \omega) + (1 - \omega)^2 + \beta^2 - \omega^2 (\alpha^2 + \beta^2) \\
&= (1 - \omega^2)(\alpha^2 + \beta^2) - 2\alpha (1 - \omega) + (1 - \omega)^2 \\
&= (1 - \omega)\left[(\alpha - 1)^2 + \beta^2 + \omega (\alpha^2 + \beta^2 - 1)\right] \\
&\geq 0
\end{align*}
从而
\[
|\lambda + \omega - 1|^2 \geq |\lambda|^2 \omega^2 \geq \omega^2
\]
即
\[
|\lambda + \omega - 1| \geq |\lambda| \omega \geq \omega
\]
于是由 \( A \) 的严格对角占优或不可约对角占优性质可知 \( (\lambda + \omega - 1)D - \lambda \omega L - \omega U \) 也是严格对角占优或不可约对角占优的. 则\( (\lambda + \omega - 1)D - \lambda \omega L - \omega U \) 也是非奇异的. 
而
\begin{align*}
\det(\lambda I - L_{\omega}) &= \det\left(\lambda I - (D - \omega L)^{-1} [(1 - \omega)D + \omega U]\right) \\
&= \det\left((D - \omega L)^{-1}\right) \det\left((\lambda + \omega - 1)D - \lambda \omega L - \omega U\right) \\
&\neq 0
\end{align*}
矛盾, 因此 \( \lambda \) 不是 SOR 迭代矩阵 \( L_{\omega} \) 的特征值. 由 \( \lambda \) 的任意性可知, \( L_{\omega} \) 的特征值都满足 \( |\lambda| < 1 \), 从而 SOR 迭代收敛. 
\anothersolution{}
下仅证
\[
|\lambda + \omega - 1| \geq |\lambda| \omega \geq \omega
\]
令 \( z = \frac{1}{\lambda} \), 则 \( |z| \leq 1 \). 考虑函数
\[
f(z) = \frac{\omega}{1 - z(1 - \omega)}
\]
由于 \( 0 < 1 - \omega < 1 \), 当 \( |z| \leq 1 \) 时, \( |z(1 - \omega)| < 1 \), 故 \( f(z) \) 在闭单位圆盘 \( \overline{D(0,1)} \) 上解析.

当 \( |z| = 1 \) 时, 有
\[
|1 - z(1 - \omega)| \geq |1 - (1 - \omega)| = \omega
\]
因此
\[
|f(z)| = \left| \frac{\omega}{1 - z(1 - \omega)} \right| \leq 1
\]
由最大模原理, 在 \( |z| \leq 1 \) 上恒有 \( |f(z)| \leq 1 \). 特别地, 当 \( 0 < |z| \leq 1 \) 时,
\[
\left| \frac{\omega}{1 - z(1 - \omega)} \right| \leq 1 \quad \Rightarrow \quad |1 - z(1 - \omega)| \geq \omega
\]
代入 \( z = \frac{1}{\lambda} \), 得
\[
\left| 1 - \frac{1}{\lambda}(1 - \omega) \right| \geq \omega \quad \Rightarrow \quad |\lambda - (1 - \omega)| \geq |\lambda| \omega
\]
证毕.
\end{solution}


% =============== 题14 ===============
\question{}证明: 若\(A\)为具有正对角元的实对称矩阵, 则 JOR 方法收敛的充分必要条件是 \(A\)与\(2\omega^{-1}D-A\)均为正定对称矩阵. 
其中, JOR 迭代为\[ x_{k+1} = x_k - \omega D^{-1} (Ax_k - b) \]
或者
\[ x_{k+1} = (I - \omega D^{-1} A)x_k + \omega D^{-1} b \]
\begin{solution}
$D = \operatorname{diag}(a_{11}, \ldots, a_{nn})$ 的对角元均为正数, 所以记
\[
B_{\omega} = I - \omega D^{-1}A = D^{-\frac{1}{2}}(I - \omega D^{-\frac{1}{2}}AD^{-\frac{1}{2}})D^{\frac{1}{2}}. 
\]

即$B_{\omega}$ 和 $I - \omega D^{-\frac{1}{2}}AD^{-\frac{1}{2}}$ 相似, 因此它们有相同的特征值. 因为 $A$ 是实对称矩阵, 所以 $I - \omega D^{-\frac{1}{2}}AD^{-\frac{1}{2}}$ 也是实对称矩阵, 
则有\(B_{\omega}\)和\(I - \omega D^{-\frac{1}{2}}AD^{-\frac{1}{2}}\)的特征值相同且都为实数. 

\begin{enumerate}
\item \textbf{必要性:}
设 JOR 方法收敛, 即$\rho(B_{\omega}) < 1$, 所以 $I - \omega D^{-\frac{1}{2}}AD^{-\frac{1}{2}}$ 的任一特征值 $\alpha$ 都满足 $-1 < \alpha < 1$. 
于是考虑矩阵
\[
I - (I - \omega D^{-\frac{1}{2}}AD^{-\frac{1}{2}}) = \omega D^{-\frac{1}{2}}AD^{-\frac{1}{2}}
\]

的任一特征值满足 $1 - \alpha > 0$, 所以 $\omega D^{-\frac{1}{2}}AD^{-\frac{1}{2}}$正定, 
又 $ \omega > 0 $ (一般如此规定), 故矩阵 $A$ 正定. 再考虑
\[
I + (I - \omega D^{-\frac{1}{2}}AD^{-\frac{1}{2}}) = 2I - \omega D^{-\frac{1}{2}}AD^{-\frac{1}{2}} = \omega D^{-\frac{1}{2}}(2\omega^{-1}D - A)D^{-\frac{1}{2}}
\]

的任一特征值满足 $1 + \alpha > 0$, 所以 $2I - \omega D^{-\frac{1}{2}}AD^{-\frac{1}{2}}$正定, 
同样有矩阵 $2\omega^{-1}D - A$ 也正定. 

\item \textbf{充分性:}
因为 $A$ 正定, 且
\[
A = \omega^{-1}D(I - B_{\omega}) = \omega^{-1} D^{\frac{1}{2}} D^{\frac{1}{2}}(I - B_{\omega} )D^{-\frac{1}{2}}D^{\frac{1}{2}}  
\]

所以 $I - B_{\omega}$ 和 $A$ 一样是正定矩阵, 任一特征值都大于 $0$, 故 $B_{\omega}$ 的特征值均小于 $1$. 又因为 $2\omega^{-1}D - A$ 正定, 且
\[
2\omega^{-1}D - A = \omega^{-1}D(I + B_{\omega}) = \omega^{-1} D^{\frac{1}{2}} D^{\frac{1}{2}}(I + B_{\omega} )D^{-\frac{1}{2}}D^{\frac{1}{2}}    
\]

所以 $I + B_{\omega}$ 正定, 任一特征值都大于 $0$, 故 $B_{\omega}$ 的特征值均大于 $-1$. 因此 $\rho(B_{\omega}) < 1$, 即有 JOR 方法收敛. 
\end{enumerate}
\anothersolution{}
可以参考题8直接证明等价性.
\end{solution}

% =============== 题15 ===============
\question{}设 $x_k$ 由最速下降法产生, 证明:
\[
\phi(x_k) \leq \left[1 - \frac{1}{\kappa_2(A)}\right] \phi(x_{k-1})
\]
其中 $\kappa_2(A) = \|A\|_2 \|A^{-1}\|_2$. 
\begin{solution}
有 $A$ 对称正定且$\kappa_2(A) = \|A\|_2 \|A^{-1}\|_2 > 0$. 又最速下降法的递推公式:
\[
\phi(x_k) = \phi(x_{k-1}) - \frac{(r_{k-1}^{\T} p_{k-1})^2}{p_{k-1}^{\T} A p_{k-1}} = \phi(x_{k-1}) - \frac{(r_{k-1}^{\T} r_{k-1})^2}{r_{k-1}^{\T} A r_{k-1}}
\]

\begin{enumerate}
    \item 当 $\phi(x_{k-1}) \leq 0$ 时, 
    由最速下降法的性质知 $\phi(x_k) < \phi(x_{k-1}) \leq 0$, 于是
    \[
    \phi(x_k) < \phi(x_{k-1}) \le \left[1 - \frac{1}{\kappa_2(A)}\right] \phi(x_{k-1})
    \]
    不等式成立. 

    \item 当 $\phi(x_{k-1}) > 0$ 时, 
    由于$A^{-1}$ 正定, 注意到有
    \[
    r_{k-1}^{\T} A^{-1} r_{k-1} = (b - A x_{k-1})^{\T} A^{-1} (b - A x_{k-1}) = \phi(x_{k-1}) + b^{\T} A^{-1} b \geq \phi(x_{k-1}) > 0
    \]
    代入递推公式, 有
    \begin{align*}
    \phi(x_k) &= \phi(x_{k-1}) \left[ 1 - \frac{(r_{k-1}^{\T} r_{k-1})^2}{\phi(x_{k-1}) \cdot (r_{k-1}^{\T} A r_{k-1})} \right] \\
    &\leq \phi(x_{k-1}) \left[ 1 - \frac{(r_{k-1}^{\T} r_{k-1}) \cdot (r_{k-1}^{\T} r_{k-1})}{(r_{k-1}^{\T} A^{-1} r_{k-1}) \cdot (r_{k-1}^{\T} A r_{k-1})} \right]
    \end{align*}
    又由 Cauchy-Schwarz 不等式, 有
    \[
    r_{k-1}^{\T} A r_{k-1} \leq \|r_{k-1}\|_2 \|A r_{k-1}\|_2 \leq \|A\|_2 \|r_{k-1}\|_2^2 = \|A\|_2 \cdot r_{k-1}^{\T} r_{k-1}
    \]
    且
    \[
    r_{k-1}^{\T} A^{-1} r_{k-1} \leq \|r_{k-1}\|_2 \|A^{-1} r_{k-1}\|_2 \leq \|A^{-1}\|_2 \|r_{k-1}\|_2^2 = \|A^{-1}\|_2 \cdot r_{k-1}^{\T} r_{k-1}
    \]
    代入得
    \[
    \phi(x_k) \leq \phi(x_{k-1}) \left[ 1 - \frac{1}{\|A\|_2 \|A^{-1}\|_2} \right] = \phi(x_{k-1}) \left[ 1 - \frac{1}{\kappa_2(A)} \right]
    \]
\end{enumerate}
综上, 不等式 $\phi(x_k) \leq \left[1 - \frac{1}{\kappa_2(A)}\right] \phi(x_{k-1})$ 得证. 

\end{solution}


% =============== 题16 ===============  
\question{}证明: 当最速下降法在有限步求得极小值时, 最后一步迭代的下降方向必是 \( A \) 的一个特征向量. 

\begin{solution}
假定在 \( k+1 \) 步迭代后, 得到了精确解 \( x_{k+1} = x_* \), 即
\[ x_* = x_k + \frac{r_k^{\T} r_k}{r_k^{\T} A r_k} r_k \]
从而有
\[ b = A x_* = A x_k + \frac{r_k^{\T} r_k}{r_k^{\T} A r_k} A r_k \]
记:
\[ \lambda = \frac{r_k^{\T} A r_k}{r_k^{\T} r_k} \]
又 $ r_k = b - A x_k $, 整理可得
\[ A r_k = \lambda r_k \]
\end{solution}

% =============== 题17 ===============  
\question{}
 \( A \in \R^{n \times n} \) 对称正定, \( p_1, \ldots, p_k \in \R^n \) 满足 \( p_i^{\T} A p_j = 0 \), \( i \neq j \). 证明 \( p_1, \ldots, p_k \) 线性无关. 
\begin{solution}
设有 \( \alpha_1, \ldots, \alpha_k \) 满足
\[
\alpha_1 p_1 + \cdots + \alpha_k p_k = 0
\]
则对一切 \( i = 1, \ldots, k \), 上式左乘\(p_i^{\T} A\)有
\[
0 = p_i^{\T} A (\alpha_1 p_1 + \cdots + \alpha_k p_k) = \alpha_i p_i^{\T} A p_i
\]
由于\(p_i \ne 0\), 有\( p_i^{\T} A p_i \neq 0 \), 由此得出 \( \alpha_i = 0, \forall i = 1, \ldots, k \). 因此, \( p_1, p_2, \ldots, p_k \) 线性无关. 
\end{solution}

% =============== 题18 ===============  
\question{}设 \( A \) 是一个只有 \( k \) 个互不相同特征值的 \( n \times n \) 实对称矩阵, \( r \) 是任一 \( n \) 维实向量. 证明: 子空间 \( \operatorname{span}\{r, Ar, \ldots, A^{n-1}r\} \) 的维数至多是 \( k \). 

\begin{solution}
设 \( A \) 的 \( n \) 个线性无关的特征向量为 \( x_1, x_2, \ldots, x_n \), 对应的特征值为 \( \lambda_1, \lambda_2, \ldots, \lambda_n \). 又设 \( r = k_1x_1 + k_2x_2 + \cdots + k_nx_n \). 则
\[
A^l r = k_1 \lambda_1^l x_1 + k_2 \lambda_2^l x_2 + \cdots + k_n \lambda_n^l x_n \quad (l = 0, 1, 2, \ldots)
\]
考察矩阵 \([r, Ar, \ldots, A^{n-1}r] = [k_1x_1, k_2x_2, \ldots, k_nx_n] B\), 其中 \( B \) 恰是范德蒙德矩阵: 
\[
B = \begin{bmatrix} 
1 & \lambda_1 & \cdots & \lambda_1^{n-1} \\ 
1 & \lambda_2 & \cdots & \lambda_2^{n-1} \\ 
\vdots & \vdots & \ddots & \vdots \\ 
1 & \lambda_n & \cdots & \lambda_n^{n-1} 
\end{bmatrix}
\]
由于 \( \lambda_1, \lambda_2, \ldots, \lambda_n \) 中只有 \( k \) 个不同的值, 于是 \( \operatorname{rank}(B) = k \), 从而 \( \operatorname{rank}[r, Ar, \ldots, A^{n-1}r] \leq \operatorname{rank}(B) = k \). 

\anothersolution{}由于 \( A \) 是实对称矩阵, 因此 \( A \) 可对角化, 其对应最小多项式无重根, 即有
\[
m(\lambda) = (\lambda - \lambda_1)(\lambda - \lambda_2) \cdots (\lambda - \lambda_k)
\]
其中 \( \lambda_1, \lambda_2, \ldots, \lambda_k \) 是 \( A \) 的特征值. 因此有
\[
A^k + c_{k-1} A^{k-1} + \cdots + c_1 A + c_0 I = 0
\]
其中 \( c_{k-1}, \ldots, c_0 \) 为常数. \\
于是\( A^k \)可以被\(r, Ar, \ldots, A^{k-1}r\)线性表出, 则子空间 \[ \operatorname{span}\{r, Ar, \ldots, A^{k - 1}r, A^{k}r\} = \operatorname{span}\{r, Ar, \ldots, A^{k - 1}r\} \triangleq S \]
类似地, 我们可以递推(在$A$满足的等式两边同时左乘或者右乘矩阵\(A\))得到 \[ \operatorname{span}\{r, Ar, \ldots, A^{n-1}r\} = \operatorname{span}\{r, Ar, \ldots, A^{k - 1}r\},\: \forall n \geq k \]
又由于\( S \)的维数至多为 \( k \), 
故 \( \operatorname{span}\{r, Ar, \ldots, A^{n-1}r\} \) 的维数至多为 \( k \). 

\end{solution}

% =============== 题19 ===============  
\question{}证明: 如果系数矩阵 \( A \) 至多有 \( l \) 个互不相同的特征值, 则共轭梯度法至多 \( l \) 步就可得到方程组 \( Ax = b \) 的精确解. 

\begin{solution}
由题 \( 18 \) 可知 Krylov 子空间 \( \operatorname{span}\{r, Ar, \ldots, A^{n-1}r \} \) 的维数最多是 \( l \) , 所以共轭梯度法至多 \( l \) 步便可得到方程组 \( Ax = b \) 的精确解. 
\end{solution}

% =============== 题20 ===============  
\question{}
证明: 用共轭梯度法求得的 \( x_k \) 有如下的误差估计: 
\[
\|x_k - x_*\|_2 \leq 2\sqrt{\kappa_2} \left( \frac{\sqrt{\kappa_2} - 1}{\sqrt{\kappa_2} + 1} \right)^k \|x_0 - x_*\|_2
\]
其中 \(\kappa_2 = \kappa_2(A) = \|A\|_2 \|A^{-1}\|_2\). 

\begin{solution}
由课本定理可知
\[
\|x_k - x_*\|_A \leq 2 \left( \frac{\sqrt{\kappa_2} - 1}{\sqrt{\kappa_2} + 1} \right)^k \|x_0 - x_*\|_A
\]
故只需证明
\[
\frac{\|x_k - x_*\|_2}{\|x_0 - x_*\|_2} \leq \sqrt{\kappa_2} \frac{\|x_k - x_*\|_A}{\|x_0 - x_*\|_A}
\]
记
\[
\alpha = x_k - x_*, \quad \beta = x_0 - x_*
\]
则只需证
\[
(\alpha^{\T}\alpha)(\beta^{\T} A\beta) \leq \|A\|_2 \|A^{-1}\|_2 (\beta^{\T}\beta)(\alpha^{\T} A\alpha)
\]

事实上, 对于任一正定矩阵$M$都有
\[
\| M \|_2^2 = (\sqrt{\| M^{\T}M \|_2})^2 = \| M^2 \|_2
\]
由\(A\)的正定性可知$A^{1/2}$和$A^{-1/2}$也是正定的(正定矩阵的平方根分解), 于是 
\[
\alpha^{\T}\alpha = \|\alpha\|_2^2
= \|A^{1/2}A^{-1/2}\alpha\|_2^2
\leq \|A^{-1/2}\|_2^2 \|A^{1/2}\alpha\|_2^2
= \|A^{-1}\|_2 (\alpha^{\T} A\alpha)
\]
同样有 
\[
\beta^{\T} A\beta = \|A^{1/2}\beta\|_2^2
\leq \|A^{1/2}\|_2^2 \|\beta\|_2^2
= \|A\|_2 (\beta^{\T}\beta)
\]
上两式相乘即为所需, 证毕. 
\end{solution}


% =============== 题21 ===============          
\question{}设 \( A \in \R^{n \times n} \) 是对称正定的, \( \mathcal{X} \) 是 \( \R^n \) 的一个 \( k \) 维子空间. 证明:   
\( x_k \in \mathcal{X} \)  
满足  
\[ \| x_k - A^{-1} b \|_A = \min_{x \in \mathcal{X}} \| x - A^{-1} b \|_A \]  
的充分必要条件是  
\[ r_k = b - A x_k \]  
垂直于子空间 \( \mathcal{X} \), 其中 \( b \in \R^n \) 是任意给定的. 

\begin{solution}
直接从充分性和必要性两方面分别证即可. 

\textbf{充分性:}
当\( r_k \perp \mathcal{X} \), 即
\[
(b - A x_k)^{\T} x = 0 \Leftrightarrow b^{\T}x = x_k^{\T} A x, \forall x \in \mathcal{X}
\]
和
\[
(b - A x_k)^{\T} x_k = 0 \Leftrightarrow b^{\T}x_k = x_k^{\T} A x_k
\]
时, 要证明
\begin{align*}
\| x_k - A^{-1} b \|_A = \min_{x \in \mathcal{X}} \| x - A^{-1} b \|_A &\Leftrightarrow \| x_k - A^{-1} b \|_A \leq \| x - A^{-1} b \|_A   \\
&\Leftrightarrow x_k^{\T} A x_k - 2 b^{\T} x_k \leq x^{\T} A x - 2 b^{\T} x \\
\end{align*}
只需证 \( \forall x \in \mathcal{X} \), 恒成立
\[
x^{\T} A x - 2 x_k^{\T} A x + x_k^{\T} A x_k \geq 0
\]
然而显然有
\[
x^{\T} A x - 2 x_k^{\T} A x + x_k^{\T} A x_k = \|x - x_k\|_{A}^2 \geq 0
\]
则充分性得证. 

\textbf{必要性:}
当
\[
\| x_k - A^{-1} b \|_A = \min_{x \in \mathcal{X}} \| x - A^{-1} b \|_A
\]
时, \( \forall x \in \mathcal{X} \), \( \forall \lambda \in \R \)有
\[
\| x_k - A^{-1} b \|_A \leq \| (x_k + \lambda x) - A^{-1} b \|_A 
\]
即
\[
x_k^{\T} A x_k - 2 b^{\T} x_k \leq (x_k + \lambda x)^{\T} A (x_k + \lambda x) - 2 b^{\T} (x_k + \lambda x)
\]
于是
\[
(x^{\T} A x) \lambda^2 + 2 (x^{\T} A x_k - b^{\T} x) \lambda \geq 0
\]
考虑其对应一元二次方程的判别式
\[
\Delta = 4 (x^{\T} A x_k- b^{\T} x)^2 - 4 \cdot (x^{\T} A x) \cdot 0 \leq 0
\]  
即
\[
x^{\T}(Ax_k-b) = 0, \forall x \in \mathcal{X} \Leftrightarrow r_k \perp \mathcal{X}
\]
必要性得证. 

\anothersolution{}必要性同上, 下证充分性: 注意到 \( \forall x \in \mathcal{X} \), 有
\begin{align*}
\| x - A^{-1} b \|_A^2 &= \| x_k - A^{-1} b + x - x_k \|_A^2   \\
&= \| x_k - A^{-1} b \|_A^2 + \| x - x_k \|_A^2 + 2 \langle x_k - A^{-1} b, x - x_k\rangle_A  \\
&= \| x_k - A^{-1} b \|_A^2 + \| x - x_k \|_A^2 + 2 \langle Ax_k - b, x - x_k\rangle  \\
&\geq \| x_k - A^{-1} b \|_A^2
\end{align*}
即原命题得证. 

\textbf{再另解: }可以考虑内积空间中的投影定理.
\end{solution}



\end{questions}

\end{document}